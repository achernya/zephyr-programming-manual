% Copyright (c) 1988 Massachusetts Institute of Technology
%	$Source: /tmp/git-rcsimport-Ge1GfW1M7y/rcs/input-queue.tex,v $
%	$Author: jtkohl $
%	$Header: /tmp/git-rcsimport-Ge1GfW1M7y/rcs/input-queue.tex,v 2.0 1989/04/05 15:52:38 jtkohl Exp $
%
\subsection{Using the Input Queue}
\label{input-queue}

The \Zephyr\ library keeps a queue of notices as they are received.
When a packet arrives, it is placed into the queue.  If the packet is
part of a fragmented notice, the notice is reassembled as the pieces
arrive.  When all fragments have been received, the count of complete
messages is incremented.  If no fragments of an incomplete notice arrive
during a fixed, short interval after the last fragment was received, all the
stored fragments of that notice are discarded.  In this way resources
are reclaimed if a failure occurs.

The following functions allow an application to test the status of the
input queue.

\subsubsection{ZPending}
\label{ZPending}

\template{int}{ZPending}{}
\etemplate
\prereq{ZInitialize, ZOpenPort}
\errors{UNIX errors, ZERR_MAXQLEN, ZERR_EOF}

The ZPending function places any newly received packets into the input
queue, and then returns the total number of complete notices available
in the queue.  The value -1 is returned if an error occurs, with the
error code placed in the global variable {\bf errno}.

\subsubsection{ZQLength}
\label{ZQLength}

\template{int}{ZQLength}{}
\etemplate
\prereq{ZInitialize, ZOpenPort}
\errors{None}

The ZQLength function returns the total number of complete notices in
the input queue.  No new packets are placed into the input queue.

\subsubsection{ZPeekNotice}
\label{ZPeekNotice}

\template{Code_t}{ZPeekNotice}{notice, from}
\tline{ZNotice_t}{*notice}
\tline{struct sockaddr_in}{*from}
\etemplate
\prereq{ZInitialize, ZOpenPort}
\errors{UNIX errors, ZERR_NOPORT, ZERR_PKTLEN, ZERR_QLEN, ZERR_BADPKT,
ZERR_VERS}

The ZPeekNotice function ``peeks at'' the next notice in the input
queue, and returns that notice.  It functions identically to
ZReceiveNotice (\myref{ZReceiveNotice}), except that the notice is left
in the input queue.

After a successful return from the ZPeekNotice function,
ZFreeNotice (\myref{ZFreeNotice}) should be called with argument {\bf
notice} when the caller has finished using {\bf notice}.

\subsubsection{ZPeekIfNotice}
\label{ZPeekIfNotice}

\template{Code_t}{ZPeekIfNotice}{notice, from, predicate, args}
\tline{ZNotice_t}{*notice}
\tline{struct sockaddr_in}{*from}
\tline{int}{(*predicate)()}
\tline{char}{*args}
\etemplate
\prereq{ZInitialize, ZOpenPort}
\errors{UNIX errors, ZERR_BADPKT, ZERR_VERS, ZERR_PKTLEN, ZERR_NOPORT,
ZERR_QLEN}

The ZPeekIfNotice function acts identically to the ZCheckIfNotice function
(\myref{ZCheckIfNotice}), except that the notice is left in
the input queue.

After a successful return from the ZPeekIfNotice function,
ZFreeNotice (\myref{ZFreeNotice}) should be called with argument {\bf
notice} when the caller has finished using {\bf notice}.
