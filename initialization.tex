% Copyright (c) 1988 Massachusetts Institute of Technology
%	$Source: /tmp/git-rcsimport-Ge1GfW1M7y/rcs/initialization.tex,v $
%	$Author: jtkohl $
%	$Header: /tmp/git-rcsimport-Ge1GfW1M7y/rcs/initialization.tex,v 2.0 1989/04/05 15:52:35 jtkohl Exp $
%
\subsection{Initialization}
\label{initialization}

\subsubsection{ZInitialize}
\label{ZInitialize}

\template{Code_t}{ZInitialize}{}
\etemplate
\prereq{None}
\errors{ZERR_HMPORT}

The \Zephyr\ library must be initialized before it can be used.
ZInitialize performs this function.  It caches information such as the
user's username, the host's name, and the HostManager's port number to
speed up future operations, and initializes other internal state
variables.  If the port for the HostManager is not found in
\filename{/etc/services}, ZERR_HMPORT is returned.

\subsubsection{ZOpenPort}
\label{ZOpenPort}

\template{Code_t}{ZOpenPort}{port}
\tline{unsigned short}{*port}
\etemplate
\prereq{ZInitialize}
\errors{UNIX errors, ZERR_PORTINUSE}

Before the \Zephyr\ library can perform any operations that require
the sending or receiving of notices, a {\em port\/} must be allocated
for these transactions.  The ZOpenPort function accomplishes this.  It
should be called before any of the functions which send or receive
notices.  There are three ways to call ZOpenPort:

\begin{itemize}
\item{With {\bf port} NULL:}
A port is allocated randomly.
\item{With {\bf port} pointing to an integer containing 0:}
A port is allocated randomly, and {\bf *port} is set to the port
number.
\item{With {\bf port} pointing to a a non-zero integer:}
The port {\bf *port} is allocated, if possible.  If {\bf *port} cannot
be allocated, ZERR_PORTINUSE is returned.
\end{itemize}
