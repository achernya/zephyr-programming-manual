% Copyright (c) 1988 Massachusetts Institute of Technology
%	$Source: /tmp/git-rcsimport-Ge1GfW1M7y/rcs/receiving-acks.tex,v $
%	$Author: jtkohl $
%	$Header: /tmp/git-rcsimport-Ge1GfW1M7y/rcs/receiving-acks.tex,v 2.0 1989/04/05 15:55:48 jtkohl Exp $
%
\subsection{Receiving Acknowledgments}
\label{receiving-acknowledgments}

As mentioned in previous sections (\myref{overview} and
\myref{notice-kinds}) different kinds of notices trigger various
levels of acknowledgment.  The acknowledgment from the HostManager to
the sending client is handled internally.  However, any other
acknowledgments that have been requested must be handled by the
application program.  These will normally consist solely of
acknowledgments from the server.

The structure of the returning acknowledgment is described in
\myref{acknowledgment-structure}.  Acknowledgments have the same
unique ID as the original notice.  Thus, it is desirable to have some
way to check or wait for incoming notices with known characteristics
(such as a certain unique ID).  A simple application may perform the
following sequence of tasks repeatedly:

\begin{itemize}
\item Send out a notice
\item Wait for a server acknowledgment for that notice
\item Take action based on the information contained in the
acknowledgment
\item Repeat
\end{itemize}
The following functions make this a painless process.

\subsubsection{Using Predicates}
\label{using-predicates}

A {\em predicate\/} is a C function which takes a series of arguments
and returns a true or false value indicating some relationship (or
lack of relationship) between the arguments.  All predicates which can
be used by the functions described below must take the format:

\nwsimpletemplate{int}{{\em predicate\/}}{notice, arg}
\nwtline{ZNotice_t}{*notice}
\nwtline{char}{*arg}
\nwetemplate

\noindent The function should return 1 if the {\bf notice} satisfies
some prespecified relationship with {\bf arg}, 0 otherwise.

\paragraph{ZCompareUIDPred and ZCompareMultiUIDPred}
\label{ZCompareUIDPred}
\label{ZCompareMultiUIDPred}

The only predicates supplied by the \Zephyr\ library are ZCompareUIDPred
and ZCompareMultiUIDPred.  Both take a notice and a pointer to a
ZUnique_Id_t as arguments.  ZCompareUIDPred returns 1 if the unique ID
of the notice is the same as the supplied unique ID, or 0 if they are
different.  ZCompareMultiUIDPred returns 1 if the z_multiuid field of
the notice is the same as the supplied unique ID, or 0 if they are
different.

\subsubsection{ZCheckIfNotice}
\label{ZCheckIfNotice}

\template{Code_t}{ZCheckIfNotice}{notice, from, predicate, args}
\tline{ZNotice_t}{*notice}
\tline{struct sockaddr_in}{*from}
\tline{int}{(*predicate)()}
\tline{char}{*args}
\etemplate
\prereq{ZInitialize, ZOpenPort}
\errors{UNIX errors, ZERR_BADPKT, ZERR_VERS, ZERR_PKTLEN, ZERR_NOPORT,
ZERR_QLEN, ZERR_NONOTICE}

The ZCheckIfNotice function scans the notice input queue.  For each
notice, it calls the {\bf predicate} function with the notice and {\bf
arg}.  If the {\bf predicate} returns 1, ZCheckIfNotice allocates a
packet buffer and copies the notice data into this buffer, then parses
the packet into {\bf notice}, removes the notice from the queue, and
returns ZERR_NONE.  If no notice is found which is accepted by the {\bf
predicate}, ZCheckIfNotice returns ZERR_NONOTICE.  {\bf *from} is filled
in with the address of the host which sent the notice.

After a successful return from the ZCheckIfNotice function,
ZFreeNotice (\myref{ZFreeNotice}) should be called with argument {\bf
notice} when the caller has finished using {\bf notice}.

\subsubsection{ZFreeNotice}
\label{ZFreeNotice}

\template{void}{ZFreeNotice}{notice}
\tline{ZNotice_t}{*notice}
\etemplate
\prereq{None}
\errors{None}

ZFreeNotice frees up the storage allocated for a notice by other library
routines.  ZFreeNotice should only be used for notices returned by
library functions whose documentation instructs the programmer to use
ZFreeNotice.

\subsubsection{ZIfNotice}
\label{ZIfNotice}

\template{Code_t}{ZIfNotice}{notice, from, predicate, args}
\tline{ZNotice_t}{*notice}
\tline{struct sockaddr_in}{*from}
\tline{int}{(*predicate)()}
\tline{char}{*args}
\etemplate
\prereq{ZInitialize, ZOpenPort}
\errors{UNIX errors, ZERR_BADPKT, ZERR_VERS, ZERR_PKTLEN, ZERR_NOPORT,
ZERR_QLEN}

The ZIfNotice function performs identically to the ZCheckIfNotice
function above, except that if there is no notice in the input queue
that fits the desired criteria as determined by {\bf predicate},
ZIfNotice will wait until a satisfying notice is received, and then
perform as described above for ZCheckIfNotice.

After a successful return from the ZIfNotice function,
ZFreeNotice (\myref{ZFreeNotice}) should be called with argument {\bf
notice} when the caller has finished using {\bf notice}.

\subsubsection{ZCompareUID}
\label{ZCompareUID}

\template{int}{ZCompareUID}{uid1, uid2}
\tline{ZUnique_Id_t}{*uid1}
\tline{ZUnique_Id_t}{*uid2}
\etemplate
\prereq{None}
\errors{None}

There are times when the above routines may be inefficient.  For
example, an application may simply want to accept incoming notices,
and decide on the fly whether a notice is an acknowledgment to a
previously sent notice.  The ZCompareUID function provides an easy
means to determine if two unique ID's are the same.  It returns 1 if
they are the same, and 0 if they are different.

\subsubsection{Sample Application}

This application demonstrates use of ZSendNotice and ZIfNotice by
sending a simple message to a username specified on the command line.  It
verifies that a username was specified, initializes the library, fills in
the notice, sends it, and then waits for an acknowledgment.  When the
acknowledgment is received, it checks the response and prints a message
indicating the response type.

\begin{code}
#include <zephyr/zephyr.h>

main(argc,argv)
    int argc;
    char *argv[];
{
    ZNotice_t notice, retnotice;
    Code_t retval;

    /* verify username is specified */
    if (argc < 2) {
        fprintf(stderr, "No username specified!\n");
        exit (1);
    }

    /* Initialize the library */
    if ((retval = ZInitialize()) != ZERR_NONE) {
        com_err("sample", retval, "while initializing");
        exit (1);
    }

    bzero((char *)&notice, sizeof(notice));
    /* initialize the notice after zeroing the entire structure */
    /* Fill in the notice fields */
    notice.z_kind = ACKED;      /* ACKED since we want an acknowledgment */
    notice.z_port = 0;  /* Will be filled in by HostManager */
    notice.z_class = "MESSAGE";
    notice.z_class_inst = "PERSONAL";
    notice.z_opcode = "";
    /* The sender will get filled in by the library if 0 */
    notice.z_sender = (char *)NULL;
    /* Send to the person named on the command line */
    notice.z_recipient = argv[1];
    notice.z_default_format = "";

    notice.z_message = "Hi there!\n";
    /* Make sure we include the trailing NULL in the length */
    notice.z_message_len = strlen(notice.z_message)+1;

    /* Send the notice unauthenticated */
    /* ZSendNotice will automatically open a port for us */
    if ((retval = ZSendNotice(&notice, ZNOAUTH)) != ZERR_NONE) {
        com_err("sample", retval, "while sending notice");
        exit (1);
    }

    /* Wait for the server response. */
    if ((retval = ZIfNotice(&retnotice, (struct sockaddr_in *)0,
                            ZCompareUIDPred,
                            (char *)&notice.z_uid)) != ZERR_NONE) {
            com_err("sample", retval, "while waiting for ack");
            exit (1);
    } 

    /* Was there an error? */
    if (retnotice.z_kind == SERVNAK) {
        printf("Received authentication failure while sending\n");
        exit (1);
    }

    /* Check the message body for the acknowledgment information */
    if (!strcmp(retnotice.z_message, ZSRVACK_SENT))
        printf("Message sent!\n");
    else
        printf("%s not receiving messages!\n", argv[1]);

    ZFreeNotice(&retnotice);
    exit (0);
}
\end{code}
