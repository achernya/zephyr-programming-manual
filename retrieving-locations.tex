% Copyright (c) 1988 Massachusetts Institute of Technology
%	$Source: /tmp/git-rcsimport-Ge1GfW1M7y/rcs/retrieving-locations.tex,v $
%	$Author: raeburn $
%	$Id: retrieving-locations.tex,v 2.1 1990/07/15 15:53:25 raeburn Exp $
%
\subsection{Retrieving User Locations}
\label{retrieving-locations}

The following functions allow an application to retrieve information
from the user location service (\myref{user-location}).

\subsubsection{ZLocateUser}
\label{ZNewLocateUser}

\template{Code_t}{ZLocateUser}{user, nlocs, cert_routine}
\tline{char}{*user}
\tline{int}{*nlocs}
\tline{int}{(*cert_routine)()}
\etemplate
\prereq{ZInitialize}
\errors{Kerberos errors, UNIX errors, ZERR_PKTLEN, ZERR_ILLVAL,
ZERR_HMDEAD, ZERR_BADPKT, ZERR_VERS, ZERR_QLEN}

The ZNewLocateUser function queries the user location service for the
location of user {\bf user}.  The locations are stored in allocated
storage.  The number of locations retrieved is returned in {\bf
*nlocs}.  If another ZNewLocateUser call is done, the old storage is
freed and new storage allocated as necessary.  {\bf cert_routine} is a
function that will authenticate the query, or NULL if no such
authentication is requested.  See \myref{authenticated-notices} for a
description of authentication.

An error return of ZERR_VERS usually indicates a version mismatch of the
following types:
\begin{itemize}
\item A new application running on a host with an old HostManager.
\item An old server.
\end{itemize}
The {\tt zstat} program will display the version numbers of both the
HostManager and server.  Check to be sure the Protocol Version numbers
are the same as the protocol version of your application (inspect
\filename{/usr/include/zephyr/zephyr.h} to find the protocol version
numbers of your application).

\subsubsection{ZNewLocateUser}
\label{ZNewLocateUser}

Deprecated alias for ZLocateUser.

\subsubsection{ZGetLocations}
\label{ZGetLocations}

\template{Code_t}{ZGetLocations}{location, numloc}
\tline{ZLocations_t}{location[]}
\tline{int}{*numloc}
\etemplate
\prereq{ZInitialize, ZLocateUser}
\errors{ZERR_NOLOCATIONS, ZERR_NOMORELOCS}

The ZGetLocations function returns the next user location entries that
were retrieved with ZLocateUser.  If there are no stored locations,
ZERR_NOLOCATIONS is returned.  {\bf *numloc} should initially contain
the maximum number of locations that can fit in the {\bf location}
buffer.  On return, {\bf *numloc} will contain the number of entries
actually returned.  Subsequent calls to ZGetLocations will return
additional location entries if possible.  When there are no more unseen
locations in the internal storage, ZERR_NOMORELOCS is returned.  As the
pointers in the filled-in ZLocations_t structure point into private
storage used by ZLocateUser, the contents may change on future calls to
ZLocateUser.  Thus they should be copied into other storage by the
client if necessary.

\subsubsection{ZFlushLocations}
\label{ZFlushLocations}

\template{Code_t}{ZFlushLocations}{}
\etemplate
\prereq{ZInitialize, ZLocateUser}
\errors{None}

The ZFlushLocations function frees any storage allocated by the
most recent ZLocateUser call.

\subsubsection{Sample Application}

This application demonstrates use of ZLocateUser and ZGetLocations.  It
verifies that a username was specified on the command line, then
initializes the library, queries the server for the locations of the
specified user, and prints out any locations found.

\begin{code}
#include <zephyr/zephyr.h>

main(argc, argv)
    int argc;
    char *argv[];
{
    ZLocations_t location;
    Code_t retval;
    int i, totallocs, numlocs;

    /* verify a username was given on command line */
    if (argc < 2) {
        fprintf(stderr, "No username specified!\n");
        exit (1);
    }

    /* Initialize library */
    if ((retval = ZInitialize()) != ZERR_NONE) {
        com_err("sample", retval, "while initializing");
        exit (1);
    }

    /* Locate the user. */
    if ((retval = ZLocateUser(argv[1], &totallocs, ZAUTH)) != ZERR_NONE) {
        com_err("sample", retval, "while locating user");
        exit (1);
    }

    /* complain if not logged in */
    if (totallocs == 0) {
        printf("%s not logged in.\n", argv[1]);
        exit(1);
    }

    /* retrieve each location, one at a time */
    for (i=0; i<totallocs; i++) {
        numlocs = 1;
        if ((retval = ZGetLocations(&location, &numlocs)) != ZERR_NONE) {
            com_err("sample", retval, "while getting location");
            exit (1);
        }
        printf("%s at %s on %s\n", location.host, location.time,
            location.tty);
    }
    exit(0);
}
\end{code}
