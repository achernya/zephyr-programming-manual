% Copyright (c) 1988 Massachusetts Institute of Technology
%	$Source: /tmp/git-rcsimport-Ge1GfW1M7y/rcs/retrieving-subscriptions.tex,v $
%	$Author: jtkohl $
%	$Header: /tmp/git-rcsimport-Ge1GfW1M7y/rcs/retrieving-subscriptions.tex,v 2.0 1989/04/05 15:56:45 jtkohl Exp $
%
\subsection{Retrieving Subscriptions}
\label{retrieving-subscriptions}

The following functions allow an application to retrieve information
about the user's current subscriptions or the system default
subscriptions (\myref{subscription}).

\subsubsection{ZRetrieveSubscriptions}
\label{ZRetrieveSubscriptions}

\template{Code_t}{ZRetrieveSubscriptions}{port, nsubs}
\tline{unsigned short}{port}
\tline{int}{*nsubs}
\etemplate
\prereq{ZInitialize}
\errors{Kerberos errors, UNIX errors, ZERR_PKTLEN, ZERR_ILLVAL,
ZERR_HMDEAD, ZERR_BADPKT, ZERR_VERS, ZERR_QLEN}

The ZRetrieveSubscriptions function queries the server for the current
subscriptions for port {\bf port}.  The number of subscriptions
retrieved is returned in {\bf *nsubs}.  If {\bf port} is zero, the
port of the current application is substituted.  The ZGetWGPort
function (\myref{ZGetWGPort}) can be used to return the port number
associated with the user's WindowGram Client.

\subsubsection{ZRetrieveDefaultSubscriptions}
\label{ZRetrieveDefaultSubscriptions}

\template{Code_t}{ZRetrieveDefaultSubscriptions}{nsubs}
\tline{int}{*nsubs}
\etemplate
\prereq{ZInitialize}
\errors{Kerberos errors, UNIX errors, ZERR_PKTLEN, ZERR_ILLVAL,
ZERR_HMDEAD, ZERR_BADPKT, ZERR_VERS, ZERR_QLEN}

The ZRetrieveDefaultSubscriptions function queries the server for the
default subscriptions.  The number of subscriptions retrieved is
returned in {\bf *nsubs}.

\subsubsection{ZGetSubscriptions}
\label{ZGetSubscriptions}

\template{Code_t}{ZGetSubscriptions}{subscription, numsub}
\tline{ZSubscription_t}{*subscription}
\tline{int}{*numsub}
\etemplate
\prereq{ZInitialize, ZRetrieveSubscriptions or ZRetrieveDefaultSubscriptions}
\errors{ZERR_NOSUBSCRIPTIONS, ZERR_NOMORESUBSCRIPTIONS}

The ZGetSubscriptions function returns the next subscription entries
that were retrieved with ZRetrieveSubscriptions or
ZRetrieveDefaultSubscriptions.  {\bf *numsub} should initially contain
the maximum number of subscriptions that can fit in the {\bf
subscription} buffer.  On return, {\bf *numsub} will contain the
number of entries actually returned.  Subsequent calls to
ZGetSubscriptions will return additional subscription entries if
possible.  As the pointers in the ZSubscription_t structure may point
into ``private'' storage filled in by ZRetrieveSubscriptions or
ZRetrieveDefaultSubscriptions, they may changed on
future calls to these functions.  Thus they should be copied into other
storage by the client if necessary.

\subsubsection{ZFlushSubscriptions}
\label{ZFlushSubscriptions}

\template{Code_t}{ZFlushSubscriptions}{}
\etemplate
\prereq{ZInitialize, ZRetrieveSubscriptions or ZRetrieveDefaultSubscriptions}
\errors{None}

The ZFlushSubscriptions function frees any storage allocated by
the ZRetrieveSubscriptions or ZRetrieveDefaultSubscriptions functions.
This will be automatically performed one of these functions is called again.

\subsubsection{Sample Application}

This application demonstrates use of ZGeWGPort, ZRetrieveSubscriptions
and ZRetrieveDefaultSubscriptions.  It initializes the library,
retrieves the WindowGram port (printing an error if it is not able to
retrieve the port number), retrieves and prints the subscriptions, and
finally retrieves and prints the default subscriptions.

\begin{code}
#include <zephyr/zephyr.h>

main()
{
    ZSubscription_t subscription;
    Code_t retval;
    int wgport, totalsubs;

    /* Initialize the library */
    if ((retval = ZInitialize()) != ZERR_NONE) {
        com_err("sample", retval, "while initializing");
        exit (1);
    }

    wgport = ZGetWGPort();

    if (wgport == -1)
        printf("Can't retrieve current subscriptions\n");
    else {
        /* retrieve WindowGram's subscriptions */
        if ((retval = ZRetrieveSubscriptions((unsigned short)wgport,
            &totalsubs)) != ZERR_NONE) {
               com_err("sample", retval, "while retrieving subscriptions");
               exit (1);
        }
        printf("Your current subscriptions:\n");
        print_subs(totalsubs);
    }

    /* retrieve default subscriptions */
    if ((retval = ZRetrieveDefaultSubscriptions(&totalsubs)) != ZERR_NONE) {
        com_err("sample", retval,
            "while retrieving default subscriptions");
        exit (1);
    }
    printf("Default subscriptions:\n");
    print_subs(totalsubs);
    exit(0);
}

print_subs(totalsubs)
    int totalsubs;
{
    ZSubscription_t subscription;
    Code_t retval;
    int i, numsubs;

    for (i=0; i<totalsubs; i++) {
        numsubs = 1;
        if ((retval = ZGetSubscriptions(&subscription,
            &numsubs)) != ZERR_NONE) {
                com_err("sample", retval, "while getting subscription");
                exit (1);
        }
        printf("<%s,%s,%s>\n", subscription.class, subscription.classinst,
            subscription.recipient);
    }
}

\end{code}
