% Copyright (c) 1988 Massachusetts Institute of Technology
%       $Source: /tmp/git-rcsimport-Ge1GfW1M7y/rcs/ex-zstat.tex,v $
%       $Author: jtkohl $
%       $Header: /tmp/git-rcsimport-Ge1GfW1M7y/rcs/ex-zstat.tex,v 2.0 1989/04/05 15:51:33 jtkohl Exp $
%
\subsection{zstat}
\label{ex-zstat}

\begin{code}
/* This file is part of the Project Athena Zephyr Notification System.
 * It contains the zstat program.
 *
 *      Created by:     David C. Jedlinsky
 *
 *      Source: /mit/zephyr/src/clients/zstat/RCS/zstat.c,v 
 *      Author: jtkohl 
 *
 *      Copyright (c) 1987,1988 by the Massachusetts Institute of Technology.
 *      For copying and distribution information, see the file
 *      "mit-copyright.h". 
 */

#include <zephyr/zephyr.h>
#include "../server/zserver.h"
#include <sys/param.h>
#include <sys/socket.h>
#include <netdb.h>
#include <stdio.h>
#include <string.h>

#ifndef lint
#ifndef SABER
static char rcsid_zstat_c[] =
 "Header: zstat.c,v 1.6 88/06/28 10:42:54 jtkohl Exp ";
#endif SABER
#endif lint
                     
extern long atol();

char *hm_head[] = { "Current server =",
                     "Items in queue:",
                     "Client packets received:",
                     "Server packets received:",
                     "Server changes:",
                     "Version:",
                     "Looking for a new server:",
                     "Time running:",
                     "Size:",
                     "Machine type:"
};
#define HM_SIZE (sizeof(hm_head) / sizeof (char *))
char *srv_head[] = { 
        "Current server version =",
        "Packets handled:",
        "Uptime:",
        "Server states:",
};
#define SRV_SIZE        (sizeof(srv_head) / sizeof (char *))

int serveronly = 0,hmonly = 0;
int outoftime = 0;
u_short hm_port,srv_port;

main(argc, argv)
        int argc;
        char *argv[];
{
        Code_t ret;
        char hostname[MAXHOSTNAMELEN];
        int optchar;
        struct servent *sp;
        extern char *optarg;
        extern int optind;

        if ((ret = ZInitialize()) != ZERR_NONE) {
                com_err("zstat", ret, "initializing");
                exit(-1);
        }

        if ((ret = ZOpenPort((u_short *)0)) != ZERR_NONE) {
                com_err("zstat", ret, "opening port");
                exit(-1);
        }

        while ((optchar = getopt(argc, argv, "sh")) != EOF) {
                switch(optchar) {
                case 's':
                        serveronly++;
                        break;
                case 'h':
                        hmonly++;
                        break;
                case '?':
                default:
                        usage(argv[0]);
                        exit(1);
                }
        }

        if (serveronly && hmonly) {
                fprintf(stderr,"Only one of -s and -h may be specified\n");
                exit(1);
        }

        if (!(sp = getservbyname("zephyr-hm","udp"))) {
                fprintf(stderr,"zephyr-hm/udp: unknown service\n");
                exit(-1);
        }

        hm_port = sp->s_port;

        if (!(sp = getservbyname("zephyr-clt","udp"))) {
                fprintf(stderr,"zephyr-clt/udp: unknown service\n");
                exit(-1);
        }

        srv_port = sp->s_port;

        if (optind == argc) {
                if (gethostname(hostname, MAXHOSTNAMELEN) < 0) {
                        com_err("zstat",errno,"while finding hostname");
                        exit(-1);
                }
                do_stat(hostname);
                exit(0);
        }

        for (;optind<argc;optind++)
                do_stat(argv[optind]);

        exit(0);
}

do_stat(host)
        char *host;
{
        char srv_host[MAXHOSTNAMELEN];
        
        if (serveronly) {
                (void) srv_stat(host);
                return;
        }

        if (hm_stat(host,srv_host))
                return;

        if (!hmonly)
                (void) srv_stat(srv_host);
}

hm_stat(host,server)
        char *host,*server;
{
        char *line[20],*mp;
        int sock,i,nf,ret;
        struct hostent *hp;
        struct sockaddr_in sin;
        long runtime;
        struct tm *tim;
        ZNotice_t notice;
        extern int timeout();
        
        bzero((char *)&sin,sizeof(struct sockaddr_in));

        sin.sin_port = hm_port;

        if ((sock = socket(PF_INET, SOCK_DGRAM, 0)) < 0) {
                perror("socket:");
                exit(-1);
        }
        
        sin.sin_family = AF_INET;

        if ((hp = gethostbyname(host)) == NULL) {
                fprintf(stderr,"Unknown host: %s\n",host);
                exit(-1);
        }
        bcopy(hp->h_addr, (char *) &sin.sin_addr, hp->h_length);

        printf("Hostmanager stats: %s\n",hp->h_name);
        
        (void) bzero((char *)&notice, sizeof(notice));
        notice.z_kind = STAT;
        notice.z_port = 0;
        notice.z_class = HM_STAT_CLASS;
        notice.z_class_inst = HM_STAT_CLIENT;
        notice.z_opcode = HM_GIMMESTATS;
        notice.z_sender = "";
        notice.z_recipient = "";
        notice.z_default_format = "";
        notice.z_message_len = 0;
        
        if ((ret = ZSetDestAddr(&sin)) != ZERR_NONE) {
                com_err("zstat", ret, "setting destination");
                exit(-1);
        }
        if ((ret = ZSendNotice(&notice, ZNOAUTH)) != ZERR_NONE) {
                com_err("zstat", ret, "sending notice");
                exit(-1);
        }

        (void) signal(SIGALRM,timeout);
        outoftime = 0;
        (void) alarm(10);
        if (((ret = ZReceiveNotice(&notice, (struct sockaddr_in *) 0))
             != ZERR_NONE) &&
            ret != EINTR) {
                com_err("zstat", ret, "receiving notice");
                return (1);
        }
        (void) alarm(0);
        if (outoftime) {
                fprintf(stderr,"No response after 10 seconds.\n");
                return (1);
        }
        
        mp = notice.z_message;
        for (nf=0;mp<notice.z_message+notice.z_message_len;nf++) {
                line[nf] = mp;
                mp += strlen(mp)+1;
        }

        (void) strcpy(server,line[0]);

        printf("HostManager protocol version = %s\n",notice.z_version);

        for (i=0; (i < nf) && (i < HM_SIZE); i++) {
                if (!strncmp("Time",hm_head[i],4)) {
                        runtime = atol(line[i]);
                        tim = gmtime(&runtime);
                        printf("%s %d days, %02d:%02d:%02d\n", hm_head[i],
                                tim->tm_yday,
                                tim->tm_hour,
                                tim->tm_min,
                                tim->tm_sec);
                }
                else
                        printf("%s %s\n",hm_head[i],line[i]);
        }

        printf("\n");
        
        (void) close(sock);
        ZFreeNotice(&notice);
        return(0);
}

srv_stat(host)
        char *host;
{
        char *line[20],*mp;
        int sock,i,nf,ret;
        struct hostent *hp;
        struct sockaddr_in sin;
        ZNotice_t notice;
        long runtime;
        struct tm *tim;
        extern int timeout();
        
        bzero((char *) &sin,sizeof(struct sockaddr_in));

        sin.sin_port = srv_port;

        if ((sock = socket(PF_INET, SOCK_DGRAM, 0)) < 0) {
                perror("socket:");
                exit(-1);
        }
        
        sin.sin_family = AF_INET;

        if ((hp = gethostbyname(host)) == NULL) {
                fprintf(stderr,"Unknown host: %s\n",host);
                exit(-1);
        }
        bcopy(hp->h_addr, (char *) &sin.sin_addr, hp->h_length);

        printf("Server stats: %s\n",hp->h_name);
        
        (void) bzero((char *)&notice, sizeof(notice));
        notice.z_kind = UNSAFE;
        notice.z_port = 0;
        notice.z_class = ZEPHYR_ADMIN_CLASS;
        notice.z_class_inst = "";
        notice.z_opcode = ADMIN_STATUS;
        notice.z_sender = "";
        notice.z_recipient = "";
        notice.z_default_format = "";
        notice.z_message_len = 0;
        
        if ((ret = ZSetDestAddr(&sin)) != ZERR_NONE) {
                com_err("zstat", ret, "setting destination");
                exit(-1);
        }
        if ((ret = ZSendNotice(&notice, ZNOAUTH)) != ZERR_NONE) {
                com_err("zstat", ret, "sending notice");
                exit(-1);
        }

        (void) signal(SIGALRM,timeout);
        outoftime = 0;
        (void) alarm(10);
        if (((ret = ZReceiveNotice(&notice, (struct sockaddr_in *) 0))
            != ZERR_NONE) &&
            ret != EINTR) {
                com_err("zstat", ret, "receiving notice");
                return (1);
        }
        (void) alarm(0);
        if (outoftime) {
                fprintf(stderr,"No response after 10 seconds.\n");
                return (1);
        } 
        
        mp = notice.z_message;
        for (nf=0;mp<notice.z_message+notice.z_message_len;nf++) {
                line[nf] = mp;
                mp += strlen(mp)+1;
        }

        printf("Server protocol version = %s\n",notice.z_version);
        
        for (i=0; i < nf; i++) {
                if (i < 2)
                        printf("%s %s\n",srv_head[i],line[i]);
                else if (i == 2) { /* uptime field */
                        runtime = atol(line[i]);
                        tim = gmtime(&runtime);
                        printf("%s %d days, %02d:%02d:%02d\n",
                               srv_head[i],
                               tim->tm_yday,
                               tim->tm_hour,
                               tim->tm_min,
                               tim->tm_sec);
                } else if (i == 3) {
                        printf("%s\n",srv_head[i]);
                        printf("%s\n",line[i]);
                } else printf("%s\n",line[i]);
        }
        printf("\n");
        
        (void) close(sock);
        return(0);
}

usage(s)
        char *s;
{
        fprintf(stderr,"usage: %s [-s] [-h] [host ...]\n",s);
        exit(1);
}

timeout()
{
        outoftime = 1;
}
\end{code}
