% Copyright (c) 1988 Massachusetts Institute of Technology
%	$Source: /tmp/git-rcsimport-Ge1GfW1M7y/rcs/raw-notices.tex,v $
%	$Author: jtkohl $
%	$Header: /tmp/git-rcsimport-Ge1GfW1M7y/rcs/raw-notices.tex,v 2.0 1989/04/05 15:55:43 jtkohl Exp $
%
\subsection{Using Raw Notices}
\label{raw-notices}

It is occasionally useful to be able to transmit packets without
having the library fill in missing information for you.  For example,
the library will usually initialize the {\bf z_uid} and the
authentication fields for you.  However, if you wanted to send a
message with a specific unique ID, as is the case with an
acknowledgment, you would not want the library to interfere.  The
following routines allow an application to handle notices in their
``raw'', or untouched, form.

\subsubsection{ZFormatRawNotice}
\label{ZFormatRawNotice}

\template{Code_t}{ZFormatRawNotice}{notice, buffer, ret_len}
\tline{ZNotice_t}{*notice}
\tline{char}{**buffer}
\tline{int}{*ret_len}
\etemplate
\prereq{ZInitialize}
\errors{ZERR_PKTLEN, ZERR_HEADERLEN}

The ZFormatRawNotice function performs identically to the
ZFormatNotice function (\myref{ZFormatNotice}), except that no part of
the notice is changed when it is formatted.  The notice in {\bf
notice} is formatted into a packet and placed in a newly allocated
buffer. The address of this buffer is returned in {\bf *buffer}.
The length of the resulting packet is returned in {\bf *ret_len}.

After a successful call to ZFormatRawNotice, the standard C library routine
free() should be called with argument {\bf *buffer} when the caller is
finished using the packet.

\subsubsection{ZFormatSmallRawNotice}
\label{ZFormatSmallRawNotice}

\template{Code_t}{ZFormatSmallRawNotice}{notice, buffer, ret_len}
\tline{ZNotice_t}{*notice}
\tline{ZPacket_t}{buffer}
\tline{int}{*ret_len}
\etemplate
\prereq{ZInitialize}
\errors{ZERR_PKTLEN, ZERR_HEADERLEN}

The ZFormatSmallRawNotice function performs identically to the
ZFormatRawNotice function above, except that no packet fragmentation is
allowed, and if the notice will not fit into a single ZPacket_t buffer,
ZERR_PKTLEN is returned.  The caller must provide the packet buffer in
{\bf buffer}.

\subsubsection{ZFormatRawNoticeList}
\label{ZFormatRawNoticeList}

\template{Code_t}{ZFormatRawNoticeList}{notice, list, nitems, buffer,
ret_len}
\tline{ZNotice_t}{*notice}
\tline{char}{*list[]}
\tline{int}{nitems}
\tline{char}{**buffer}
\tline{int}{*ret_len}
\etemplate
\prereq{ZInitialize}
\errors{ZERR_PKTLEN, ZERR_HEADERLEN}

The ZFormatRawNoticeList function performs identically to the
ZFormatNoticeList function (\myref{ZFormatNoticeList}), except that no
part of the notice is changed when it is formatted.  The notice in {\bf
notice} is formatted into a packet and placed in a newly allocated
buffer. The address of this buffer is returned in {\bf *buffer}.  The
body of the notice is take from the {\bf nitems} items in {\bf list},
which are concatenated together with a NULL following each item.  The
length of the resulting packet is returned in {\bf *ret_len}.

After a successful call to ZFormatRawNoticeList, the standard C library routine
free() should be called with argument {\bf *buffer} when the caller is
finished using the packet.

\subsubsection{ZFormatSmallRawNoticeList}
\label{ZFormatSmallRawNoticeList}

\template{Code_t}{ZFormatSmallRawNoticeList}{notice, list, nitems, buffer,
ret_len}
\tline{ZNotice_t}{*notice}
\tline{char}{*list[]}
\tline{int}{nitems}
\tline{ZPacket_t}{buffer}
\tline{int}{*ret_len}
\etemplate
\prereq{ZInitialize}
\errors{ZERR_PKTLEN, ZERR_HEADERLEN}

The ZFormatSmallRawNoticeList function performs identically to the
ZFormatRawNoticeList function above, except that no packet fragmentation is
allowed, and if the notice will not fit into a single ZPacket_t buffer,
ZERR_PKTLEN is returned.  The caller must provide the packet
buffer in {\bf buffer}.

\subsubsection{ZSendRawNotice}
\label{ZSendRawNotice}

\template{Code_t}{ZSendRawNotice}{notice}
\tline{ZNotice_t}{*notice}
\etemplate
\prereq{ZInitialize}
\errors{UNIX errors, ZERR_PKTLEN, ZERR_HMDEAD, ZERR_BADPKT, ZERR_VERS,
ZERR_QLEN}

The ZSendRawNotice function performs identically to the ZSendNotice
function (\myref{ZSendNotice}), except that no part of the notice is
changed when it is formatted, and no authentication is performed.

\subsubsection{ZSendRawList}
\label{ZSendRawList}

\template{Code_t}{ZSendRawList}{notice, list, nitems}
\tline{ZNotice_t}{*notice}
\tline{char}{*list[]}
\tline{int}{nitems}
\etemplate
\prereq{ZInitialize}
\errors{UNIX errors, ZERR_PKTLEN, ZERR_HMDEAD, ZERR_BADPKT, ZERR_VERS,
ZERR_QLEN}

The ZSendRawList function performs identically to the ZSendList
function (\myref{ZSendList}), except that no part of the notice is
changed when it is formatted, and no authentication is performed.



