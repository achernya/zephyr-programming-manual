% Copyright (c) 1988 Massachusetts Institute of Technology
%	$Source: /tmp/git-rcsimport-Ge1GfW1M7y/rcs/rendezvous.tex,v $
%	$Author: jtkohl $
%	$Header: /tmp/git-rcsimport-Ge1GfW1M7y/rcs/rendezvous.tex,v 2.0 1989/04/05 15:56:37 jtkohl Exp $
%
\subsection{Using \Zephyr\ as a Rendezvous Service}
\label{rendezvous-service}

Occasionally there are times when an application needs \Zephyr's
ability to find and contact users, but needs to send the user a large
quantity of information instead of a single message.\footnote{A good
example of this is Project Athena's On-Line Consulting System, which
needs to send entire conversations between users.}  In this case,
\Zephyr's notice sending functions (ZSendNotice, {\em etc.\/}) prove
inefficient because of the overhead involved in packet fragmentation and
reassembly, as well as an acknowledgement scheme designed for simple,
single-packet transactions.  Instead of using \Zephyr\ for the
transmission of the information, it can be used as a ``rendezvous
service.''

In this case, \Zephyr's notice sending capabilities can be used to
transmit a host and port number (as part of the message body in a
notice).  Once these are received, a more efficient TCP/IP connection
can be established for the actual data transmission.  The IPC tutorial
(\cite{ipc-tutorial}) gives an introduction describing how to establish
a TCP/IP connection under 4.3BSD.

