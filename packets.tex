% Copyright (c) 1988 Massachusetts Institute of Technology
%	$Source: /tmp/git-rcsimport-Ge1GfW1M7y/rcs/packets.tex,v $
%	$Author: jtkohl $
%	$Header: /tmp/git-rcsimport-Ge1GfW1M7y/rcs/packets.tex,v 2.0 1989/04/05 15:55:13 jtkohl Exp $
%
\subsection{Using Packets}
\label{using-packets}

While notices, in the guise of ZNotice_t structures, are the objects
most frequently manipulated by applications, another representation,
the ``packet'', is available.  A packet is a single buffer which
contains all of the information of a ZNotice_t structure.  The packet
is what is transmitted and received over the network, and is
broken down into its constituent fields to become a ZNotice_t
structure.

The following functions are provided for applications that want to
manipulate or store packets directly.  This may be useful since
packets are generally easier to store than ZNotice_t structures.\footnote{The
strings in a ZNotice_t structure always contain pointers into the
original packet; thus they must stay together.  However, a packet can
stand on its own and be parsed into a ZNotice_t structure when
necessary.}

See \myref{sending-notices} for a description of what fields in a notice
need to be initialized before using the following routines.

\subsubsection{ZFormatNotice}
\label{ZFormatNotice}

\template{Code_t}{ZFormatNotice}{notice, buffer, ret_len, cert_routine}
\tline{ZNotice_t}{*notice}
\tline{char}{**buffer}
\tline{int}{*ret_len}
\tline{int}{(*cert_routine)()}
\etemplate
\prereq{ZInitialize}
\errors{Kerberos errors, UNIX errors, ZERR_PKTLEN, ZERR_ILLVAL}

The ZFormatNotice function takes {\bf notice} and formats it into a
packet.  The packet is stored in a newly allocated buffer.  The address of
this buffer is returned in {\bf *buffer}.  The length of the resulting
packet is returned in {\bf *ret_len}.  {\bf cert_routine} is called, if
non-NULL, as described in \myref{authenticated-notices}, to authenticate
the notice and place authentication information into the packet.

After a successful call to ZFormatNotice, the standard C library routine
free() should be called with argument {\bf *buffer} when the caller is
finished using the packet.

\subsubsection{ZFormatNoticeList}
\label{ZFormatNoticeList}

\template{Code_t}{ZFormatNoticeList}{notice, list, nitems, buffer,
ret_len, cert_routine}
\tline{ZNotice_t}{*notice}
\tline{char}{*list[]}
\tline{int}{nitems}
\tline{char}{**buffer}
\tline{int}{*ret_len}
\tline{int}{(*cert_routine)()}
\etemplate
\prereq{ZInitialize}
\errors{Kerberos errors, UNIX errors, ZERR_PKTLEN, ZERR_ILLVAL}

The ZFormatNoticeList function is a hybrid between the ZFormatNotice
function (\myref{ZFormatNotice}) and the ZSendList function
(\myref{ZSendList}).  It formats the notice into a newly allocated packet like
ZFormatNotice, but the message body comes from the {\bf nitems} fields
of {\bf list}.  The fields are concatenated together with a NULL after
each field.

After a successful call to ZFormatNoticeList, the standard C library routine
free() should be called with argument {\bf *buffer} when the caller is
finished using the packet.

\subsubsection{ZSendPacket}
\label{ZSendPacket}

\template{Code_t}{ZSendPacket}{packet, len, waitforack}
\tline{char}{*packet}
\tline{int}{len}
\tline{int}{waitforack}
\etemplate
\prereq{ZInitialize}
\errors{UNIX errors, ZERR_ILLVAL, ZERR_HMDEAD, ZERR_BADPKT, ZERR_VERS,
ZERR_PKTLEN, ZERR_QLEN}

The ZSendPacket function simply transmits the {\bf len} byte long
packet in {\bf packet} to the current destination address.  If the
{\bf waitforack} argument is non-zero, it 
will wait until it receives a HostManager acknowledgment before returning.

\subsubsection{ZReceivePacket}
\label{ZReceivePacket}

\template{Code_t}{ZReceivePacket}{buffer, ret_len, from}
\tline{ZPacket_t}{buffer}
\tline{int}{*ret_len}
\tline{struct sockaddr_in}{*from}
\etemplate
\prereq{ZInitialize, ZOpenPort}
\errors{UNIX errors, ZERR_NOPORT, ZERR_PKTLEN, ZERR_QLEN}

The ZReceivePacket function reads and removes the next complete packet from the
input queue and returns it in {\bf buffer}, which should be the standard
Z_MAXPKTLEN bytes long.  The actual size of the received packet is returned in
{\bf *ret_len}, and the address from which the packet was received is
returned in {\bf *from}.  If the next complete packet is larger than
Z_MAXPKTLEN, ZERR_PKTLEN is returned.
If you need to use the packet representation of a notice larger than
Z_MAXPTKLEN, use ZReceiveNotice (\myref{ZReceiveNotice}) and access the
{\bf z_packet} field of the notice.

\subsubsection{ZPeekPacket}
\label{ZPeekPacket}

\template{Code_t}{ZPeekPacket}{buffer, ret_len, from}
\tline{char}{**buffer}
\tline{int}{*ret_len}
\tline{struct sockaddr_in}{*from}
\etemplate
\prereq{ZInitialize, ZOpenPort}
\errors{UNIX errors, ZERR_NOPORT, ZERR_PKTLEN, ZERR_QLEN, ZERR_BADPKT,
ZERR_VERS}

The ZPeekPacket function ``peeks at'' the next packet in the input
queue.  The packet is placed in a newly allocated buffer.  The address
of the buffer is returned in {\bf *buffer}.  The size of the received
packet is returned in {\bf *ret_len}, and the address from which the
packet was received is returned in {\bf *from}.  The packet is not
removed from the input queue.

After a successful call to ZPeekPacket, the standard C library routine
free() should be called with argument {\bf *buffer} when the caller is
finished using the packet.

\subsubsection{ZParseNotice}
\label{ZParseNotice}

\template{Code_t}{ZParseNotice}{buffer, buffer_len, notice}
\tline{char}{*buffer}
\tline{int}{buffer_len}
\tline{ZNotice_t}{*notice}
\etemplate
\prereq{ZInitialize}
\errors{ZERR_BADPKT, ZERR_VERS}

The ZParseNotice function breaks a packet down into its constituent
fields, and builds a ZNotice_t structure.  {\bf buffer} contains the
{\bf buffer_len} byte long packet to be parsed.  The resulting notice is
returned in {\bf *notice}.

\subsubsection{Sample Application}

This application demonstrates use of ZFormatNotice and ZParseNotice.  It
initializes the library, fills in a notice, formats it, prints the
formatted representation, parses the notice, prints the parsed
notice, and frees the allocated storage.

\begin{code}
#include <zephyr/zephyr.h>

main()
{
    ZNotice_t notice, newnotice;
    char *buffer;
    int buffer_len;
    int retval;

    /* Initialize the library */
    if ((retval = ZInitialize()) != ZERR_NONE) {
        com_err("sample", retval, "while initializing");
        exit(1);
    }

    bzero((char *)&notice, sizeof(notice));
    /* initialize the notice after zeroing the entire structure */
    notice.z_kind = ACKED;
    notice.z_port = 0;  /* Will be filled in by HostManager */
    notice.z_class = "MESSAGE";
    notice.z_class_inst = "PERSONAL";
    notice.z_opcode = "";
    /* The sender will get filled in by the library if 0 */
    notice.z_sender = (char *)NULL;
    notice.z_recipient = ZGetSender(); /* myself */
    notice.z_default_format = "This is the default format\n$1";

    notice.z_message = "Hello - This is an example!";
    /* Make sure we include the trailing NULL in the length */
    notice.z_message_len = strlen(notice.z_message)+1;

    /* Format the notice */
    if ((retval = ZFormatNotice(&notice, &buffer, &buffer_len, ZNOAUTH))
        != ZERR_NONE) {
        com_err("sample", retval, "while formatting notice");
        exit(1);
    }

    /* print the formatted packet.
       use write() here so that the entire buffer is printed.  printf() would
       stop at the first NULL. 
       Unfortunately, however, the NULL's are normally invisible unless the
       output is piped through /bin/cat -v */
    write(1, buffer, buffer_len);

    /* Parse the notice */
    if ((retval = ZParseNotice(buffer, buffer_len, &newnotice)) != ZERR_NONE) {
        com_err("sample", retval, "while parsing notice");
        exit(1);
    }
    /* Print the fields in the notice */
    printf("\nversion = '%s'\nz_time = %d,%d\nz_port = %u\nz_auth = %d\n\
z_authent_len = %d\nz_ascii_authent = '%s'\nz_class = '%s'\n\
z_class_inst = '%s'\nz_opcode = '%s'\nz_sender = '%s'\nz_recipient = '%s'\n\
z_default_format = '%s'\nz_multinotice = '%s'\nz_checksum = %lu\n\
z_num_other_fields = %d\nz_other_fields[0] = '%s'\nz_other_fields[1] = '%s'\n\
z_other_fields[2] = '%s'\nz_other_fields[3] = '%s'\n\
z_other_fields[4] = '%s'\nz_other_fields[5] = '%s'\n\
z_other_fields[6] = '%s'\nz_other_fields[7] = '%s'\n\
z_other_fields[8] = '%s'\nz_other_fields[9] = '%s'\n\
z_message_len = %d\nz_message = '%s'\n",
           newnotice.z_version,
           newnotice.z_time.tv_sec, newnotice.z_time.tv_usec,
           newnotice.z_port,
           newnotice.z_auth,
           newnotice.z_authent_len,
           newnotice.z_ascii_authent,
           newnotice.z_class,
           newnotice.z_class_inst,
           newnotice.z_opcode,
           newnotice.z_sender,
           newnotice.z_recipient,
           newnotice.z_default_format,
           newnotice.z_multinotice,
           newnotice.z_checksum,
           newnotice.z_num_other_fields,
           newnotice.z_other_fields[0],
           newnotice.z_other_fields[1],
           newnotice.z_other_fields[2],
           newnotice.z_other_fields[3],
           newnotice.z_other_fields[4],
           newnotice.z_other_fields[5],
           newnotice.z_other_fields[6],
           newnotice.z_other_fields[7],
           newnotice.z_other_fields[8],
           newnotice.z_other_fields[9],
           newnotice.z_message_len,
           newnotice.z_message);
    /* Free storage allocated by ZFormatNotice */
    free(buffer);
    exit(0);
}
\end{code}
