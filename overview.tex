% Copyright (c) 1988 Massachusetts Institute of Technology
%	$Source: /tmp/git-rcsimport-Ge1GfW1M7y/rcs/overview.tex,v $
%	$Author: jtkohl $
%	$Header: /tmp/git-rcsimport-Ge1GfW1M7y/rcs/overview.tex,v 2.0 1989/04/05 15:53:55 jtkohl Exp $
%
\section{Overview of the Zephyr System}
\label{overview}

\subsection{Major Divisions}

The \Zephyr\ system consists of three primary sections:

\begin{itemize}
\item{\bf The \Zephyr\ clients:} These are the applications which
actually use \Zephyr\ to accomplish a task.  Examples of clients are
{\bf zwrite}, which allows a user to send messages to other users,
{\bf syslogd}, which can send system warnings to users, and {\bf zwgc},
the WindowGram client, which is the standard way for users to receive
incoming notices.  Clients are generally written using the \Zephyr\
library, which is the primary subject of this manual.

\item{\bf The HostManager:} The HostManager is the intermediary between the
clients and the rest of the \Zephyr\ system.  There is one HostManager
running on every host which supports \Zephyr\ programs.  All clients
send their outgoing notices to the HostManager.  The HostManager then
redistributes them to \Zephyr\ servers for final delivery.  The
HostManager is in charge of determining if a particular server is still
operational, and choosing a different server if necessary.

\item{\bf The \Zephyr\ servers:} These are the core of the \Zephyr\
system.  They are in charge of receiving notices from the clients (via
the HostManagers), figuring out which clients or users should receive
them, and distributing them.  They are also in charge of keeping an
up-to-date user location database.  There can be any number of servers
spread throughout a workstation environment.  The servers keep in
constant communication, sharing information about changed
subscriptions and changed user locations.
\end{itemize}

The communication between these three components is indicated
schematically in Figure~\ref{fig:zephyr-interaction}.  One server and
two hosts are shown.  Each host consists of a HostManager and two
clients.  The sending host is sending a message which is received by a
client on the receiving host.  This is a simplified
view of the computing environment.  Normally there would be hundreds
or thousands of hosts, and several servers which keep in constant
communication with each other.

\begin{figure}
% Copyright (c) 1988 Massachusetts Institute of Technology
%	$Source: /tmp/git-rcsimport-Ge1GfW1M7y/rcs/interaction-pic.tex,v $
%	$Author: jtkohl $
%	$Header: /tmp/git-rcsimport-Ge1GfW1M7y/rcs/interaction-pic.tex,v 2.0 1989/04/05 15:52:42 jtkohl Exp $
%
\setlength{\unitlength}{.1in}
% This is a hostmanager - coords are center point
\def\zhostmanager(#1,#2){
	\thinlines
	\put(#1,#2){\oval(10,5)}
	\put(#1,#2){\makebox(0,0){\small HostManager}}
}
% This is a client - coords are bottom left
\def\zclient(#1,#2){
	\thinlines
	\put(#1,#2){\framebox(10,4){\small Client}}
}
% This is a client with a dashed box - coords are bottom left
\def\zdashedclient(#1,#2){
	\thinlines
	\put(#1,#2){\dashbox{.5}(10,4){\small Client}}
}
\begin{center}
\begin{picture}(50,40)
% Draw big server at top
\thicklines
\put(15,32.5){\framebox(20,7){\large Server}}
\put(15.5,33){\framebox(19,6){}}
\put(16,33.5){\framebox(18,5){}}
% Draw hostmanager on left
\zhostmanager(15,20)
% Client under it
\zclient(10,8)
% A dashed client under that one
\zdashedclient(10,2)
% Draw hostmanager on right
\zhostmanager(35,20)
% Clients under it
\zclient(30,8)
\zdashedclient(30,2)
% Connect hostmanager on right to server
\put(37,22.5){\vector(-1,4){2.5}}
\put(37,24){\tiny notice}
\put(30.5,32.5){\vector(1,-4){2.5}}
\put(26,31){\tiny SERVACK}
% Connect client on right to hostmanager
\put(37,12){\vector(0,1){5.5}}
\put(37.5,13){\tiny notice}
\put(33,17.5){\vector(0,-1){5.5}}
\put(28,14){\shortstack{\tiny HMACK \\ \tiny \& \\ \tiny SERVACK}}
% Connect server to client on left
\put(24,32.5){\line(0,-1){23}}
\put(24,9.5){\vector(-1,0){4}}
\put(24.5,29){\tiny notice}
\put(20,11){\line(1,0){2}}
\put(22,11){\vector(0,1){21.5}}
\put(16,13){\tiny CLIENTACK}
% Label the hosts
\put(15,0){\makebox(0,0){Receiving Host}}
\put(35,0){\makebox(0,0){Sending Host}}
\end{picture}
\end{center}

\caption{Interaction between the various parts of a \Zephyr\ system.}
\label{fig:zephyr-interaction}
\end{figure}

When a notice is sent from a client, many events happen ``behind the
scenes'' that are not normally seen by the user.  However, these
events are very important to an applications programmer, and are listed
in simplified form below.  Note that many of these actions occur in
parallel so this should not be construed as an absolute order of events.

\begin{enumerate}
\item The source client program calls a \Zephyr\ library routine which
sends a notice.
\item The \Zephyr\ library sends the notice to the HostManager on the
same host, and then waits for an acknowledgment.
\item The HostManager receives the notice, and sends back an
acknowledgment (HMACK) to the originating client.
\item The \Zephyr\ library receives the acknowledgment and returns to
the calling program.
\item The HostManager forwards the notice to a \Zephyr\ server, and
appends the notice to its queue of unacknowledged notices.
\item The \Zephyr\ server receives the notice, determines its
recipients, and sends back an acknowledgment (SERVACK) to the HostManager.
\item The HostManager receives the acknowledgment, removes the notice
from its queue of unacknowledged notices, and forwards a copy of the
acknowledgment (SERVACK) to the client program.
\item The client program receives and disposes of the acknowledgment.
\item The server forwards the notice to all recipients, and
each time appends the notice to its queue of unacknowledged notices.
\item A destination client receives the notice, and sends back an
acknowledgment (CLIENTACK) to the server.
\item The server receives the acknowledgment and removes the notice
from its queue of unacknowledged notices.
\end{enumerate}

\noindent At each stage except the initial client to HostManager
communication, if an acknowledgment is not received after a certain 
length of time, the notice is retransmitted.  The initial client to
HostManager communication is not retransmitted because there is little
chance of a notice being lost when transmitted to a program on the
same machine, and thus the lack of an acknowledgment indicates that
something is very wrong.

Flags may be set in the notice to indicate how much acknowledgment
should be done.  The possibilities range from no acknowledgments to the
full acknowledgment scenario listed above.  A discussion of the
different levels of acknowledgment can be found in \myref{notice-kinds},
and a discussion of the structure of acknowledgments can be found in
\myref{acknowledgment-structure}.
