% Copyright (c) 1988 Massachusetts Institute of Technology
%	$Source: /tmp/git-rcsimport-Ge1GfW1M7y/rcs/subscriptions.tex,v $
%	$Author: jtkohl $
%	$Header: /tmp/git-rcsimport-Ge1GfW1M7y/rcs/subscriptions.tex,v 2.0 1989/04/05 15:58:43 jtkohl Exp $
%
\subsection{The Subscription Service}
\label{subscription}

Since the primary purpose of \Zephyr\ is to deliver notices from one
user or service to another, an important consideration is the means of
specifying the recipient of a notice.  An application must be able to
send a given notice to a particular recipient, a known group of
recipients, or an arbitrary, dynamically changing group of recipients.
\Zephyr\ accomplishes this by using a ``subscription'' service.  Each
\Zephyr\ notice contains three fields which determine its recipients:
{\em class\/}, {\em instance\/}, and {\em recipient\/}.  Users
subscribe to specific triples of class, instance, and recipient, as
described below, and these subscriptions are used to determine whether
or not a given user will receive a notice.

\subsubsection{The {\bf class} field}
\label{subscription-class}

The class is the top-level characteristic of a notice.  It serves two
primary purposes:

\begin{itemize}
\item It is used as the first indicator of who might be able to
receive the notice
\item It is used to determine if the sender is authorized to send a
notice of the particular class (see \myref{subscription-authorization}, 
``Subscription Authorization'').

\end{itemize}

For example, a ``MESSAGE'' class might be used to indicate a generic
user-to-user message, and a ``FILSRV'' class might be used to indicate
a file server message.

\subsubsection{The {\bf instance} field}
\label{subscription-instance}

The {\em instance\/} is a subdivision of the {\em class\/}.  Its primary
purpose is to narrow down the subject of the notice.  For example, a
notice with class ``FILSRV'' might contain the name of the fileserver
as its instance.  By itself, the instance is not very useful.  It is
simply an extra string like the class field that is used to determine
possible recipients.  However, the instance field allows wildcarding
at subscription time.  This means that a person could subscribe to
file server messages from only a particular server by specifying the
file server's name as the instance, or all file server messages by
specifying instance ``*''.

The only wildcard instance allowed is ``*''.  More complicated
regular expressions (such as ``*.MIT.EDU'' to match all hosts in the
MIT.EDU domain) are not allowed.

\subsubsection{The {\bf recipient} field}
\label{subscription-recipient}

The {\em recipient\/} is the actual username of the person the notice is
intended for.  On systems which support the Kerberos authentication
system \cite{kerberos-paper}, the recipient is the Kerberos principal
of the recipient.  A Kerberos principal is usually of the form {\em
username}@{\em realm}, where {\em realm\/} is the name of the
Kerberos realm controlling the user's host.

The recipient field may be wildcarded on both the sending and
receiving ends.  Once again ``*'' is the only valid wildcard.  These
limitations apply:

\begin{itemize}
\item If a user is subscribing to the triple \triple{\em class\/}{\em
instance}{\bf username}, where {\bf username} is the username of
the user, only notices with the user's explicit username in their
recipient field will be sent to the user.

\item If a user is subscribing to the triple \triple{\em class\/}{\em
instance}{\bf *}, only notices with a recipient of ``*'' will be
sent to the user.
\end{itemize}

Thus, if a user is subscribing to \triple{\sc message}{\sc personal}
{rfrench@\sc athena.mit.edu}, and a message is sent to
\triple{\sc message}{\sc personal}{*}, he will not receive it.  Likewise, if
a person is subscribing to
\triple{\sc message}{\sc personal}{*}, and one is sent to \triple{\sc
message}{\sc personal} {rfrench@\sc athena.mit.edu}, he will not
receive it.  Subscriptions can be combined.  Thus if a person is
subscribing to both of these triples, he will receive both messages.
Note also that a person cannot subscribe to messages destined for
users other than himself.  These limitations combine to prevent a user
from receiving another user's personal messages.

\subsubsection{Examples}
\label{subscription-examples}

A notice sent to \triple{\sc message}{\sc personal}{rfrench@\sc athena.mit.edu}
will be received by user ``rfrench'' if he subscribes to:

\ss
\begin{description}
\item \triple{\sc message}{\sc personal}{rfrench@\sc athena.mit.edu}
\item \triple{\sc message}{*}{rfrench@\sc athena.mit.edu}
\end{description}
\ess

\noindent But he would not receive it if he subscribed to:

\ss
\begin{description}
\item \triple{\sc foobar}{\sc personal}{rfrench@\sc athena.mit.edu}
\item \triple{\sc message}{\sc foobar}{rfrench@\sc athena.mit.edu}
\item \triple{\sc message}{\sc personal}{*}
\end{description}
\ess

Likewise, a notice sent to \triple{\sc filsrv}{\sc paris.mit.edu}{*}
would be received by someone subscribing to:

\ss
\begin{description}
\item \triple{\sc filsrv}{\sc paris.mit.edu}{*}
\item \triple{\sc filsrv}{*}{*}
\end{description}
\ess

\noindent But would not be received by someone subscribing to:

\ss
\begin{description}
\item \triple{\sc filsrv}{\sc paris.mit.edu}{user@\sc athena.mit.edu}
\item \triple{\sc filsrv}{*}{user@\sc athena.mit.edu}
\end{description}
\ess

\subsubsection{Subscription Authorization}
\label{subscription-authorization}

It is possible for a notice to be authenticated using the Kerberos
authentication system \cite{kerberos-paper}.  The method used to do this
is described in \myref{authenticated-notices}, ``Sending Authenticated
Notices.''  When a notice is authenticated, a
\Zephyr\ server can perform a number of tests to determine if a user is
allowed to send notices to or subscribe to messages from a particular
class.  The lists which determine these restrictions are kept on the
server machines, and may not be updated by users.  They are not
available for inspection by users.

Some of the restrictions that can be placed on a class are:

\begin{itemize}
\item Only specified users can send notices to this class
\item Only specified users can subscribe to notices from this class
\item Any notice sent to this class must be authenticated
\item Any notice sent to this class must have an instance equal to the
sender of the notice
\end{itemize}

A complete list of restrictions and how they are implemented is
available in the \Zephyr\ design document \cite{zephyr-design}.

\subsubsection{Default Subscriptions}
\label{default-subscriptions}
The Zephyr servers maintain a list of default subscriptions which are
automatically added to all subscriptions at the first subscription
request for a given port.  These subscriptions are intended to make sure
that the Operations staff can easily notify all users.  If you wish to
disable these subscriptions, see the section on ZUnsubscribeTo,
\myref{ZUnsubscribeTo}.
