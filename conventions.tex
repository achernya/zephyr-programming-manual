% Copyright (c) 1988 Massachusetts Institute of Technology
%	$Source: /tmp/git-rcsimport-Ge1GfW1M7y/rcs/conventions.tex,v $
%	$Author: jtkohl $
%	$Header: /tmp/git-rcsimport-Ge1GfW1M7y/rcs/conventions.tex,v 2.0 1989/04/05 15:50:24 jtkohl Exp $
%
\section{Manual Conventions}
\label{conventions}

The following typographical conventions are used in this manual:

\begin{itemize}
\item A combination of class, class instance, and recipient (used for
subscriptions) is written as \triple{\sc class}{\sc instance}{\sc
recipient}.

\item A function template is written as follows:

\nwtemplate{int}{function}{arg1, arg2}
\nwtline{int}{arg1}
\nwtline{char}{*arg2}
\nwetemplate
\nwprereq{Any functions that must be called before this one.}
\nwerrors{All possible error codes that could be returned.}

\item During the discussion of a function, arguments are written in
bold type, like {\bf arg1}.

\item Explicit members of a structure are also written in bold type,
like {\bf member}.

\item Filenames are written in slanted type, like \filename{filename}.

\item Symbols that are defined in an include file are written in the
normal type face, like ZERR_NONE.

\item Strings that should be entered explicitly are written between
quotes, like ``rfrench''.
\end{itemize}
