% Copyright (c) 1988 Massachusetts Institute of Technology
%	$Source: /tmp/git-rcsimport-Ge1GfW1M7y/rcs/server-funcs.tex,v $
%	$Author: jtkohl $
%	$Header: /tmp/git-rcsimport-Ge1GfW1M7y/rcs/server-funcs.tex,v 2.0 1989/04/05 15:57:49 jtkohl Exp $
%
\subsection{Server Functions}
\label{server-functions}

The following functions are normally only used by the \Zephyr\ server
and the HostManager.

\subsubsection{ZSetServerState}
\label{ZSetServerState}

\template{Code_t}{ZSetServerState}{state}
\tline{int}{state}
\etemplate
\prereq{ZInitialize}
\errors{None}

If the \Zephyr\ library is to be used to perform Server or Hostmanager
functions, the behavior of various internal routines needs to be
modified.  If {\bf state} is non-zero, the server behavior is selected.
If {\bf state} is zero, the server behavior is inhibited.

By default, the \Zephyr\ library assumes non-server behavior is desired.

\subsubsection{ZSrvSendNotice}
\label{ZSrvSendNotice}

\template{Code_t}{ZSrvSendNotice}{notice, cert_routine, send_routine}
\tline{ZNotice_t}{*notice}
\tline{int}{(*cert_routine)()}
\tline{int}{(*send_routine)()}
\etemplate
\prereq{ZInitialize}
\errors{Kerberos errors, UNIX errors, ZERR_HEADERLEN, ZERR_ILLVAL,
ZERR_HMDEAD, ZERR_BADPKT, ZERR_VERS, ZERR_QLEN}

The ZSrvSendNotice function performs identically to the ZSendNotice
function (\myref{ZSendNotice}), except that {\bf send_routine} is used
to transmit each fragment of the resulting notice.  {\bf send_routine}
will be called as follows:
\par\ss
\nwsimpletemplate{Code_t}{send_routine}{notice, buffer, len, waitforack}
\nwtline{ZNotice_t}{*notice}
\nwtline{char}{*buffer}
\nwtline{int}{len}
\nwtline{int}{waitforack}
\nwetemplate\ess
{\bf notice} points to the notice fragment corresponding to the
formatted fragment. It is provided in case {\bf send_routine} wishes to
examine fields, such as uniqe ID's, before transmitting.
{\bf buffer} points to a formatted fragment to be transmitted.
{\bf len} is the length of the fragment.
{\bf waitforack} is non-zero if the notice kind is UNACKED or ACKED and
server mode has not been set.

\subsubsection{ZSrvSendList}
\label{ZSrvSendList}

\template{Code_t}{ZSrvSendList}{notice, list, nitems, cert_routine, send_routine}
\tline{ZNotice_t}{*notice}
\tline{char}{*list[]}
\tline{int}{nitems}
\tline{int}{(*cert_routine)()}
\tline{int}{(*send_routine)()}
\etemplate
\prereq{ZInitialize}
\errors{Kerberos errors, UNIX errors, ZERR_HEADERLEN, ZERR_ILLVAL,
ZERR_HMDEAD, ZERR_BADPKT, ZERR_VERS, ZERR_QLEN}

The ZSrvSendList function performs identically to the ZSendList function
(\myref{ZSendList}), except that {\bf send_routine} is used to transmit
each fragment of the resulting notice.  {\bf send_routine} will be
called as described above for ZSrvSendNotice.

\subsubsection{ZSrvSendRawList}
\label{ZSrvSendRawList}

\template{Code_t}{ZSrvSendRawList}{notice, list, nitems, send_routine}
\tline{ZNotice_t}{*notice}
\tline{char}{*list[]}
\tline{int}{nitems}
\tline{int}{(*send_routine)()}
\etemplate
\prereq{ZInitialize}
\errors{UNIX errors, ZERR_PKTLEN, ZERR_HMDEAD, ZERR_BADPKT, ZERR_VERS,
ZERR_QLEN}

The ZSrvSendRawList function performs identically to the ZSendRawList
function (\myref{ZSendRawList}), except that {\bf send_routine} is used
to transmit each fragment of the resulting notice.  {\bf send_routine}
will be called as described above for ZSrvSendNotice.

\subsubsection{ZFormatAuthenticNotice}
\label{ZFormatAuthenticNotice}

\template{Code_t}{ZFormatAuthenticNotice}{notice, buffer, buffer_len, ret_len, session}
\tline{ZNotice_t}{*notice}
\tline{char}{*buffer}
\tline{int}{buffer_len}
\tline{int}{*ret_len}
\tline{C_Block}{session}
\etemplate
\prereq{ZInitialize}
\errors{Kerberos errors, UNIX errors, ZERR_PKTLEN, ZERR_HEADERLEN, ZERR_ILLVAL}

The ZFormatAuthenticNotice function takes {\bf notice} and formats it
into a packet.  The packet is stored in the user-supplied buffer {\bf
*buffer}.  {\bf buffer_len} should be the size of the buffer.  If the
notice will not fit in the buffer, an error is returned.  The length of
the resulting packet is returned in {\bf *ret_len}. The DES library
routine {\bf quad_cksum} is called to checksum the appropriate portions
of the notice, using {\bf session} as the session key, thus generating a
cryptographic checksum.  This checksum is stored as the checksum in the
packet.  When {\bf session} is shared between the sender and recipient
of the notice, the authentication can be verified using
ZCheckAuthentication (\myref{ZCheckAuthentication}).
