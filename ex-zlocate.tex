% Copyright (c) 1988 Massachusetts Institute of Technology
%       $Source: /tmp/git-rcsimport-Ge1GfW1M7y/rcs/ex-zlocate.tex,v $
%       $Author: jtkohl $
%       $Header: /tmp/git-rcsimport-Ge1GfW1M7y/rcs/ex-zlocate.tex,v 2.0 1989/04/05 15:50:49 jtkohl Exp $
%
\subsection{zlocate}
\label{ex-zlocate}

\begin{code}
/* This file is part of the Project Athena Zephyr Notification System.
 * It contains code for the "zlocate" command.
 *
 *      Created by:     Robert French
 *
 *      Source: /mit/zephyr/src/clients/zlocate/RCS/zlocate.c,v 
 *      Author: jtkohl 
 *
 *      Copyright (c) 1987,1988 by the Massachusetts Institute of Technology.
 *      For copying and distribution information, see the file
 *      "mit-copyright.h". 
 */

#include <zephyr/mit-copyright.h>

#include <zephyr/zephyr.h>
#include <string.h>

#ifndef lint
static char rcsid_zlocate_c[] =
 "Header: zlocate.c,v 1.8 88/08/01 14:12:14 jtkohl Exp ";
#endif lint

main(argc,argv)
        int argc;
        char *argv[];
{
        int retval,numlocs,i,one,ourargc,found;
        char *whoami,bfr[BUFSIZ],user[BUFSIZ];
        ZLocations_t locations;
        
        whoami = argv[0];

        if (argc < 2) {
                printf("Usage: %s user ... \n",whoami);
                exit(1);
        }
        
        if ((retval = ZInitialize()) != ZERR_NONE) {
                com_err(whoami,retval,"while initializing");
                exit(1);
        } 

        argv++;
        argc--;

        one = 1;
        found = 0;
        ourargc = argc;
        
        for (;argc--;argv++) {
                (void) strcpy(user,*argv);
                if (!index(user,'@')) {
                        (void) strcat(user,"@");
                        (void) strcat(user,ZGetRealm());
                } 
                if ((retval = ZLocateUser(user,&numlocs)) != ZERR_NONE) {
                        (void) sprintf(bfr,"while locating user %s",user);
                        com_err(whoami,retval,bfr);
                        continue;
                }
                if (ourargc > 1)
                        printf("\t%s:\n",user);
                if (!numlocs) {
                        printf("Hidden or not logged-in\n");
                        if (argc)
                                printf("\n");
                        continue;
                }
                for (i=0;i<numlocs;i++) {
                        if ((retval = ZGetLocations(&locations,&one))
                            != ZERR_NONE) {
                                com_err(whoami,retval,
                                        "while getting location");
                                continue;
                        }
                        if (one != 1) {
                                printf("\
%s: internal failure while getting location\n",whoami);
                                exit(1);
                        }
                        /* just use printf; make the field widths one
                         * smaller to deal with the extra separation space.
                         */
                        printf("%-*s %-*s %s\n",
                               42, locations.host,
                               7, locations.tty,
                               locations.time);
                        found++;
                }
                if (argc)
                        printf("\n");
                (void) ZFlushLocations();
        }
        if (!found)
            exit(1);
        exit(0);
}
\end{code}
