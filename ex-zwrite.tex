% Copyright (c) 1988 Massachusetts Institute of Technology
%       $Source: /tmp/git-rcsimport-Ge1GfW1M7y/rcs/ex-zwrite.tex,v $
%       $Author: jtkohl $
%       $Header: /tmp/git-rcsimport-Ge1GfW1M7y/rcs/ex-zwrite.tex,v 2.0 1989/04/05 15:51:38 jtkohl Exp $
%
\subsection{zwrite}
\label{ex-zwrite}

\begin{code}
/* This file is part of the Project Athena Zephyr Notification System.
 * It contains code for the "zwrite" command.
 *
 *      Created by:     Robert French
 *
 *      Source: /mit/zephyr/src/clients/zwrite/RCS/zwrite.c,v 
 *      Author: jtkohl 
 *
 *      Copyright (c) 1987,1988 by the Massachusetts Institute of Technology.
 *      For copying and distribution information, see the file
 *      "mit-copyright.h". 
 */

#include <zephyr/mit-copyright.h>

#include <zephyr/zephyr.h>
#include <string.h>
#include <netdb.h>

#ifndef lint
static char rcsid_zwrite_c[] =
 "Header: zwrite.c,v 1.24 88/08/01 14:13:55 jtkohl Exp ";
#endif lint

#define DEFAULT_CLASS "MESSAGE"
#define DEFAULT_INSTANCE "PERSONAL"
#define URGENT_INSTANCE "URGENT"
#define FILSRV_CLASS "FILSRV"

#define MAXRECIPS 100

int nrecips, msgarg, verbose, quiet;
char *whoami, *inst, *class, *recips[MAXRECIPS];
int (*auth)();
void un_tabify();

extern char *malloc(), *realloc();
char *fix_filsrv_inst();

main(argc, argv)
    int argc;
    char *argv[];
{
    ZNotice_t notice;
    int retval, arg, nocheck, nchars, msgsize, filsys, tabexpand;
    char bfr[BUFSIZ], *message, *signature;
    char classbfr[BUFSIZ], instbfr[BUFSIZ], sigbfr[BUFSIZ];
        
    whoami = argv[0];

    if ((retval = ZInitialize()) != ZERR_NONE) {
        com_err(whoami, retval, "while initializing");
        exit(1);
    } 

    if (argc < 2)
        usage(whoami);

    bzero((char *) &notice, sizeof(notice));

    auth = ZAUTH;
    verbose = quiet = msgarg = nrecips = nocheck = filsys = 0;
    tabexpand = 1;

    if (class = ZGetVariable("zwrite-class")) {
        (void) strcpy(classbfr, class);
        class = classbfr;
    }
    else
        class = DEFAULT_CLASS;
    if (inst = ZGetVariable("zwrite-inst")) {
        (void) strcpy(instbfr, inst);
        inst = instbfr;
    }
    else
        inst = DEFAULT_INSTANCE;
    signature = ZGetVariable("zwrite-signature");
    if (signature) {
        (void) strcpy(sigbfr, signature);
        signature = sigbfr;
    } 
        
    arg = 1;
        
    for (;arg<argc&&!msgarg;arg++) {
        if (*argv[arg] != '-') {
            recips[nrecips++] = argv[arg];
            continue;
        } 
        if (strlen(argv[arg]) > 2)
            usage(whoami);
        switch (argv[arg][1]) {
        case 'a':               /* Backwards compatibility */
            break;
        case 'o':
            class = DEFAULT_CLASS;
            inst = DEFAULT_INSTANCE;
            break;
        case 'd':
            auth = ZNOAUTH;
            break;
        case 'v':
            verbose = 1;
            break;
        case 'q':
            quiet = 1;
            break;
        case 'n':
            nocheck = 1;
            break;
        case 't':
            tabexpand = 0;
            break;
        case 'u':
            inst = URGENT_INSTANCE;
            break;
        case 'i':
            if (arg == argc-1 || filsys == 1)
                usage(whoami);
            arg++;
            inst = argv[arg];
            filsys = -1;
            break;
        case 'c':
            if (arg == argc-1 || filsys == 1)
                usage(whoami);
            arg++;
            class = argv[arg];
            filsys = -1;
            break;
        case 'f':
            if (arg == argc-1 || filsys == -1)
                usage(whoami);
            arg++;
            class = FILSRV_CLASS;
            inst = fix_filsrv_inst(argv[arg]);
            filsys = 1;
            break;
        case 'm':
            if (arg == argc-1)
                usage(whoami);
            msgarg = arg+1;
            break;
        default:
            usage(whoami);
        }
    }

    if (!nrecips && !(strcmp(class, DEFAULT_CLASS) ||
                      strcmp(inst, DEFAULT_INSTANCE))) {
        fprintf(stderr, "No recipients specified.\n");
        exit (1);
    }

    notice.z_kind = ACKED;
    notice.z_port = 0;
    notice.z_class = class;
    notice.z_class_inst = inst;
    notice.z_opcode = "PING";
    notice.z_sender = 0;
    notice.z_message_len = 0;
    notice.z_recipient = "";
    if (filsys == 1)
            notice.z_default_format = "\
@bold(Filesystem Operation Message for $instance:)\n\
From: @bold($sender)\n$message";
    else if (auth == ZAUTH)
        notice.z_default_format = "Class $class, Instance $instance:\n\
@center(To: @bold($recipient))\n$message";
    else
        notice.z_default_format =
            "@bold(UNAUTHENTIC) Class $class, Instance $instance:\n$message";

    if (!nocheck && !msgarg && filsys != 1)
        send_off(&notice, 0);
        
    if (!msgarg && isatty(0))
        printf("Type your message now.  \
End with control-D or a dot on a line by itself.\n");

    message = NULL;
    msgsize = 0;
    if (signature) {
        message = malloc((unsigned)(strlen(signature)+sizeof("From: ")+2));
        (void) strcpy(message, "From: ");
        (void) strcat(message, signature);
        msgsize = strlen(message)+1;
    }
        
    if (msgarg) {
        int size = msgsize;
        for (arg=msgarg;arg<argc;arg++)
                size += (strlen(argv[arg]) + 1);
        size++;                         /* for the newline */
        if (message)
                message = realloc(message, (unsigned) size);
        else
                message = malloc((unsigned) size);
        for (arg=msgarg;arg<argc;arg++) {
            (void) strcpy(message+msgsize, argv[arg]);
            msgsize += strlen(argv[arg]);
            if (arg != argc-1) {
                message[msgsize] = ' ';
                msgsize++;
            } 
        }
        message[msgsize] = '\n';
        message[msgsize+1] = '\0';
        msgsize += 2;
    } else {
        if (isatty(0)) {
            for (;;) {
                if (!fgets(bfr, sizeof bfr, stdin))
                    break;
                if (bfr[0] == '.' &&
                    (bfr[1] == '\n' || bfr[1] == '\0'))
                    break;
                if (message)
                        message = realloc(message,
                                          (unsigned)(msgsize+strlen(bfr)));
                else
                        message = malloc((unsigned)(msgsize+strlen(bfr)));
                (void) strcpy(message+msgsize, bfr);
                msgsize += strlen(bfr);
            }
            message = realloc(message, (unsigned)(msgsize+1));
            message[msgsize] = '\0';
        }
        else {  /* Use read so you can send binary messages... */
            while (nchars = read(fileno(stdin), bfr, sizeof bfr)) {
                if (nchars == -1) {
                    fprintf(stderr, "Read error from stdin!  Can't continue!\n");
                    exit(1);
                }
                message = realloc(message, (unsigned)(msgsize+nchars));
                bcopy(bfr, message+msgsize, nchars);
                msgsize += nchars;
            }
        } 
    }

    notice.z_opcode = "";
    if (tabexpand)
        un_tabify(&message, &msgsize);
    notice.z_message = message;
    notice.z_message_len = msgsize;

    send_off(&notice, 1);
    exit(0);
}

send_off(notice, real)
    ZNotice_t *notice;
    int real;
{
    int i, success, retval;
    char bfr[BUFSIZ];
    ZNotice_t retnotice;

    success = 0;
        
    for (i=0;i<nrecips || !nrecips;i++) {
        notice->z_recipient = nrecips?recips[i]:"";
        if (verbose && real)
            printf("Sending %smessage, class %s, instance %s, to %s\n", 
                   auth?"authenticated ":"", 
                   class, inst, 
                   nrecips?notice->z_recipient:"everyone");
        if ((retval = ZSendNotice(notice, auth)) != ZERR_NONE) {
            (void) sprintf(bfr, "while sending notice to %s", 
                    nrecips?notice->z_recipient:inst);
            com_err(whoami, retval, bfr);
            break;
        }
        if ((retval = ZIfNotice(&retnotice, (struct sockaddr_in *) 0,
                                ZCompareUIDPred, 
                                (char *)&notice->z_uid)) !=
            ZERR_NONE) {
            ZFreeNotice(&retnotice);
            (void) sprintf(bfr, "while waiting for acknowledgement for %s", 
                    nrecips?notice->z_recipient:inst);
            com_err(whoami, retval, bfr);
            continue;
        }
        if (retnotice.z_kind == SERVNAK) {
            printf("Received authorization failure while sending to %s\n", 
                   nrecips?notice->z_recipient:inst);
            ZFreeNotice(&retnotice);
            break;                      /* if auth fails, punt */
        } 
        if (retnotice.z_kind != SERVACK || !retnotice.z_message_len) {
            printf("Detected server failure while receiving \
acknowledgement for %s\n", 
                   nrecips?notice->z_recipient:inst);
            ZFreeNotice(&retnotice);
            continue;
        }
        if (!real || (!quiet && real))
            if (!strcmp(retnotice.z_message, ZSRVACK_SENT)) {
                if (real) {
                    if (verbose)
                        printf("Successful\n");
                    else
                        printf("%s: Message sent\n", 
                               nrecips?notice->z_recipient:inst);
                }
                else
                    success = 1;
            } 
            else
                if (!strcmp(retnotice.z_message, 
                            ZSRVACK_NOTSENT)) {
                    if (verbose && real) {
                        if (strcmp(class, DEFAULT_CLASS))
                            printf("Not logged in or not subscribing to\
class %s, instance %s\n", 
                                   class, inst);
                        else
                            printf("Not logged in or not subscribing to \
messages\n");
                    } 
                    else
                        if (!nrecips)
                            printf("No one subscribing to class %s, instance %s\n", 
                                   class, inst);
                        else {
                            if (strcmp(class, DEFAULT_CLASS))
                                printf("%s: Not logged in or not subscribing\
to class %s, instance %s\n", 
                                       notice->z_recipient, class, inst);
                            else
                                printf("%s: Not logged in or not subscribing \
to messages\n", 
                                       notice->z_recipient);
                        } 
                } 
                else
                    printf("Internal failure - illegal message field \
in server response\n");
        ZFreeNotice(&retnotice);
        if (!nrecips)
            break;
    }
    if (!real && !success)
        exit(1);
} 

usage(s)
    char *s;
{
    printf("Usage: %s [-a] [-d] [-v] [-q] [-u] [-o] \
[-c class] [-i inst] [-f fsname]\n\t[user ...] [-m message]\n", s);
    printf("\t-f and -c are mutually exclusive\n\
\t-f and -i are mutually exclusive\n");
    exit(1);
} 

/*
  if the -f option is specified, this routine is called to canonicalize
  an instance of the form hostname[:pack].  It turns the hostname into the
  name returned by gethostbyname(hostname)
 */

char *fix_filsrv_inst(str)
char *str;
{
        static char fsinst[BUFSIZ];
        char *ptr;
        struct hostent *hp;

        ptr = index(str,':');
        if (ptr)
                *ptr = '\0';
        
        hp = gethostbyname(str);
        if (!hp) {
                if (ptr)
                        *ptr = ':';
                return(str);
        }
        (void) strcpy(fsinst, hp->h_name);
        if (ptr) {
                (void) strcat(fsinst, ":");
                ptr++;
                (void) strcat(fsinst, ptr);
        }
        return(fsinst);
}

/* convert tabs in the buffer into appropriate # of spaces.
   slightly tricky since the buffer can have NUL's in it. */

#ifndef TABSTOP
#define TABSTOP 8                       /* #chars between tabstops */
#endif /* ! TABSTOP */

void
un_tabify(bufp, sizep)
char **bufp;
register int *sizep;
{
    register char *cp, *cp2;
    char *cp3;
    register int i;
    register int column;                /* column of next character */
    register int size = *sizep;

    for (cp = *bufp, i = 0; size; size--, cp++)
        if (*cp == '\t')
            i++;                        /* count tabs in buffer */

    if (!i)
        return;                         /* no tabs == no work */

    /* To avoid allocation churning, allocate enough extra space to convert
       every tab into TABSTOP spaces */
    /* only add (TABSTOP-1)x because we re-use the cell holding the
       tab itself */
    cp = malloc((unsigned)(*sizep + (i * (TABSTOP-1))));
    if (!cp)                            /* XXX */
        return;                         /* punt expanding if memory fails */
    cp3 = cp;
    /* Copy buffer, converting tabs to spaces as we go */
    for (cp2 = *bufp, column = 1, size = *sizep; size; cp2++, size--) {
        switch (*cp2) {
        case '\n':
        case '\0':
            /* newline or null: reset column */
            column = 1;
            *cp++ = *cp2;               /* copy the newline */
            break;
        default:
            /* copy the character */
            *cp = *cp2;
            cp++;
            column++;
            break;
        case '\t':
            /* it's a tab, compute how many spaces to expand into. */
            i = TABSTOP - ((column - 1) % TABSTOP);
            for (; i > 0; i--) {
                *cp++ = ' ';            /* fill in the spaces */
                column++;
                (*sizep)++;             /* increment the size */
            }
            (*sizep)--;                 /* remove one (we replaced the tab) */
            break;
        }
    }
    free(*bufp);                        /* free the old buf */
    *bufp = cp3;
    return;
}
\end{code}
