% Copyright (c) 1988 Massachusetts Institute of Technology
%	$Source: /tmp/git-rcsimport-Ge1GfW1M7y/rcs/receiving-notices.tex,v $
%	$Author: jtkohl $
%	$Header: /tmp/git-rcsimport-Ge1GfW1M7y/rcs/receiving-notices.tex,v 2.0 1989/04/05 15:55:53 jtkohl Exp $
%
\subsection{Receiving Notices}
\label{receiving-notices}

The following functions are used to receive and interpret incoming
notices.

\subsubsection{ZReceiveNotice}
\label{ZReceiveNotice}

\template{Code_t}{ZReceiveNotice}{notice, from}
\tline{ZNotice_t}{*notice}
\tline{struct sockaddr_in}{*from}
\etemplate
\prereq{ZInitialize, ZOpenPort}
\errors{UNIX errors, ZERR_NOPORT, ZERR_PKTLEN, ZERR_QLEN, ZERR_BADPKT,
ZERR_VERS}

The ZReceiveNotice function reads and removes the next notice from the input
queue.  If there are no notices in the queue, it blocks until a notice
arrives. A buffer to contain the notice is allocated automatically.
{\bf notice} contains the parsed packet.  All
fields of {\bf notice} are filled in as appropriate.  The address from
which the notice was received is placed in {\bf *from} if {\bf from}
is non-NULL.

After a successful return from the ZReceiveNotice function,
ZFreeNotice (\myref{ZFreeNotice}) should be called with argument {\bf
notice} when the caller has finished using {\bf notice}.

\subsubsection{Receiving Binary Data}
\label{receiving-binary}

\paragraph{ZReadAscii}
\label{ZReadAscii}

\template{Code_t}{ZReadAscii}{buffer, buffer_len, field, field_len}
\tline{char}{*buffer}
\tline{int}{buffer_len}
\tline{char}{*field}
\tline{int}{field_len}
\etemplate
\prereq{None}
\errors{ZERR_BADFIELD}

The ZReadAscii function reverses the action of the ZMakeAscii
function (\myref{ZMakeAscii}).  It takes the {\bf field_len} bytes in
{\bf field}, and converts them back into binary form in {\bf buffer}.
{\bf buffer_len} is the number of bytes available in {\bf buffer}.
ZERR_BADFIELD is returned if {\bf field} contains data in an improper
format or the data will not fit into the buffer.

\subsubsection{Receiving Authenticated Notices}
\label{receiving-authenticated}

When a notice is received, there is no way to immediately discern if
it is authenticated or not (the {\bf z_auth} field in the ZNotice_t
structure is not necessarily accurate when a notice is received).  The
ZCheckAuthentication function is provided to verify or deny the
authenticity of a notice.

\paragraph{ZCheckAuthentication}
\label{ZCheckAuthentication}

\template{Code_t}{ZCheckAuthentication}{notice, from}
\tline{ZNotice_t}{*notice}
\tline{struct sockaddr_in}{*from}
\etemplate
\prereq{ZInitialize, ZReceiveNotice}
\errors{None}

The ZCheckAuthentication function verifies the Kerberos authentication
in {\bf notice}.  It returns ZAUTH_NO if the notice is not authentic,
ZAUTH_FAILED if the notice claimed to be authentic but the authenticity
could not be verified, and ZAUTH_YES if the notice was verified to be
authentic. {\bf *from} should contain the the address from which the
notice was received.

When Kerberos is enabled, this verification uses Kerberos authentication
information.  When Kerberos is disabled, notices claiming to be
authentic cause ZCheckAuthentication to return ZAUTH_YES, and notices
not claiming to be authentic cause ZCheckAuthentication to return
ZAUTH_NO.

\subsubsection{Sample Application}

This application demonstrates use of ZOpenPort, ZSubscribeTo,
ZReceiveNotice, ZCheckAuthentication and ZFreeNotice.  It initializes
the library, opens a port, subscribes to messages, and then loops,
printing every notice it receives.  The program never terminates
normally, so it does not cancel its subscriptions.  {\bf NOTE}: any
messages which contain NULL's in them will not be printed in their
entirety, due to the way printf() handles {\bf \%s} arguments.

\begin{code}
#include <zephyr/zephyr.h>

main()
{
    ZSubscription_t sub;
    ZNotice_t notice;
    struct sockaddr_in from;
    Code_t retval;

    /* Initialize the library */
    if ((retval = ZInitialize()) != ZERR_NONE) {
        com_err("sample", retval, "while initializing");
        exit (1);
    }

    /* Open a port, so that ZSubscribeTo has a port to use */
    if ((retval = ZOpenPort((int *) 0)) != ZERR_NONE) {
        com_err("sample", retval, "while opening port");
        exit (1);
    }

    /* Subscribe this client to personal messages */

    sub.class = "MESSAGE";
    sub.classinst = "PERSONAL";
    sub.recipient = ZGetSender();

    if ((retval = ZSubscribeTo(&sub, 1, 0)) != ZERR_NONE) {
        com_err("sample", retval, "while subscribing");
        exit (1);
    }

    /* Loop and accept incoming messages and print them out */

    printf("Now accepting messages...\n");

    for (;;) {
        if ((retval = ZReceiveNotice(&notice, &from)) != ZERR_NONE) {
                com_err("sample", retval, "while receiving notice");
                exit (1);
        }
        retval = ZCheckAuthentication(&notice, &from);
        switch (retval) {
        case ZAUTH_YES:
            printf("** Authenticated message! **\n");
            break;
        case ZAUTH_NO:
            printf("** Unauthenticated message! **\n");
            break;
        case ZAUTH_FAILED:
            printf("** Forged message! **\n");
            break;
        }
        printf("Message from: %s\n", notice.z_sender);
        printf("%s\n", notice.z_message);
        ZFreeNotice(&notice);
    }

    /* We really should cancel subscriptions when we're finished,
       but this program never finishes, so... */
}
\end{code}
