% Copyright (c) 1988 Massachusetts Institute of Technology
%	$Source: /tmp/git-rcsimport-Ge1GfW1M7y/rcs/variables.tex,v $
%	$Author: jtkohl $
%	$Header: /tmp/git-rcsimport-Ge1GfW1M7y/rcs/variables.tex,v 2.0 1989/04/05 15:58:51 jtkohl Exp $
%
\subsection{Variable Handling}
\label{variables}

\subsubsection{ZGetVariable}
\label{ZGetVariable}

\template{char *}{ZGetVariable}{var}
\tline{char}{*var}
\etemplate
\prereq{None}
\errors{NULL = variable not defined}

The ZGetVariable function returns the value assigned to the variable
{\bf var}.  Variable names are case insensitive.  Two variable files are
searched: the user's private variables file
(\filename{\~{ }/.zephyr.vars})
and the system default variables file.  If the variable
is defined in the user's variables file, any system default value is
ignored.  If the variable is not defined in the user's variables file,
but is defined in the system default variables file, the system default
value is used.  If the variable is not defined in either file, NULL is
returned.  Errors encountered while opening or reading a variables file
are not returned.  The pointer returned points to storage internal to
the library.  The value should be copied before ZGetVariable is called
again.

\subsubsection{ZSetVariable}
\label{ZSetVariable}

\template{Code_t}{ZSetVariable}{var, value}
\tline{char}{*var}
\tline{char}{*value}
\etemplate
\prereq{None}
\errors{UNIX errors, ZERR_INTERNAL}

The ZSetVariable function sets the value of the variable {\bf var} to
{\bf value} in the user's private variables file.  If a variable was
already present with the same name, it is replaced.  Variable names
are case insensitive.  If the library can't find the user's home
directory, an error message is printed and ZERR_INTERNAL is returned.
A UNIX error code is returned if an error is encountered while opening,
reading, or writing the user's variables file.

\subsubsection{ZUnsetVariable}
\label{ZUnsetVariable}

\template{Code_t}{ZUnsetVariable}{var}
\tline{char}{*var}
\etemplate
\prereq{None}
\errors{UNIX errors, ZERR_INTERNAL}

The ZUnsetVariable function removes the definition of the variable
{\bf var} from the user's variables file.  If the variable is
not defined, no error is returned.  Variable names are case
insensitive.  If the library can't find the user's home directory, an
error message is printed and ZERR_INTERNAL is returned.  A UNIX error
code is returned if an error is encountered while opening, reading, or
writing the variables file.

\subsubsection{Sample Application}
\label{variables-application}

This application demonstrates use of ZSetVariable, ZUnsetVariable, and
ZGetVariable.  It plays with the ``exposure'' variable, showing
the value, unsetting any private value, showing any default value, and
restoring any private value.

\ss
\begin{code}
#include <zephyr/zephyr.h>

main()
{
    char current[100];           /* Assume the exposure is < 100 chars */
    char *value;
    Code_t status;
    int had_exposure = 0;

    /* Retrieve and copy "exposure" variable */
    if ((value = ZGetVariable("exposure")) == NULL)
        printf("No current value for 'exposure'\n");
    else {
        printf("Your value for 'exposure': %s\n", value);
        strcpy(current, value);
        had_exposure = 1;
    }

    /* Unset "exposure" variable */
    if ((status = ZUnsetVariable("exposure")) != ZERR_NONE) {
        com_err("sample", status,
                "while unsetting variable 'exposure'");
        exit(1);
    }

    /* 
     * Retrieve "exposure" variable.  If there is a system default,
     * it should be retrieved here, since exposure was unset above.
     */
    if ((value = ZGetVariable("exposure")) == NULL)
        printf("No default value for 'exposure'\n");
    else
        printf("System default value for 'exposure': %s\n", value);

    /* Re-set exposure if we saved it above */
    if (had_exposure) {
        printf("Setting 'exposure' back to original value\n");
        if ((status = ZSetVariable("exposure", current)) !=
            ZERR_NONE) {
            com_err("sample", status,
                    "while setting variable 'exposure'");
            exit(1);
        }
        /* Verify that the exposure was saved */
        if ((value = ZGetVariable("exposure")) == NULL) {
            printf("Something went really wrong here...\n");
            exit(1);
        } else
            printf("New value for 'exposure': %s\n", value);
    }
    exit(0);
}
\end{code}
\ess
