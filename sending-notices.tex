% Copyright (c) 1988 Massachusetts Institute of Technology
%	$Source: /tmp/git-rcsimport-Ge1GfW1M7y/rcs/sending-notices.tex,v $
%	$Author: jtkohl $
%	$Header: /tmp/git-rcsimport-Ge1GfW1M7y/rcs/sending-notices.tex,v 2.0 1989/04/05 15:56:50 jtkohl Exp $
%
\subsection{Sending Notices}
\label{sending-notices}

There are a number of functions that will take a notice described by a
ZNotice_t structure and transmit it to the current destination
address (usually the HostManager).  Each of the functions requires the
following fields to be initialized in the ZNotice_t structure:
\begin{list}{}{}
\item {\bf z_kind}
\item {\bf z_port} (may be 0)
\item {\bf z_class}
\item {\bf z_class_inst}
\item {\bf z_opcode}
\item {\bf z_sender} (may be 0)
\item {\bf z_recipient}
\item {\bf z_default_format}
\end{list}
The rest of the structure should be zeroed out, with code similar to
this:
\begin{code}
        ZNotice_t notice;

        bzero((char *)&notice, sizeof(notice));
        /* initialize the notice after zeroing the entire structure */
        notice.z_kind = acked;
\end{code}

In addition to the above fields, ZSendNotice requires the {\bf
z_message} and {\bf z_message_len} fields to be initialized.  The
contents of each of these fields, and the default values for those that
have them, are described in detail in the ZNotice_t section
(\myref{ZNoticet}).

\subsubsection{ZSendNotice}
\label{ZSendNotice}

\template{Code_t}{ZSendNotice}{notice, cert_routine}
\tline{ZNotice_t}{*notice}
\tline{int}{(*cert_routine)()}
\etemplate
\prereq{ZInitialize}
\errors{Kerberos errors, UNIX errors, ZERR_HEADERLEN, ZERR_ILLVAL,
ZERR_HMDEAD, ZERR_BADPKT, ZERR_VERS, ZERR_QLEN}

The ZSendNotice function takes the notice described by {\bf notice}
and sends it to the current destination address (usually the
HostManager).  {\bf cert_routine} is a function that will
authenticate the message, or is NULL if no such authentication is
requested.  See
\myref{authenticated-notices} for a description of notice
authentication.  The fields {\bf z_message} and {\bf z_message_len}
must be set in {\bf notice}.  If no port has yet been allocated,
ZOpenPort(0) (\myref{ZOpenPort}) is automatically called.

If the total header size plus the length of the message exceeds the
maximum packet size, the notice will be fragmented into multiple
packets; thus there is no limit\footnote{Well, almost no limit.
Sending very large notices ($>$ 50K) will be very inefficient using this
method.  It would be much better to use \Zephyr\ as a rendezvous
service (see \myref{rendezvous-service}).} on notice size.  This
fragmentation is transparent to the application program.

If the {\bf z_kind} field of {\bf notice} is UNACKED or ACKED, and
server mode has not been set (see ZSetServerState,
\myref{ZSetServerState}), the library
will automatically wait for an acknowledgment from the HostManager.
If the acknowledgment does not arrive after thirty seconds,
ZERR_HMDEAD is returned.

\subsubsection{ZSendList}
\label{ZSendList}

\template{Code_t}{ZSendList}{notice, list, nitems, cert_routine}
\tline{ZNotice_t}{*notice}
\tline{char}{*list[]}
\tline{int}{nitems}
\tline{int}{(*cert_routine)()}
\etemplate
\prereq{ZInitialize}
\errors{Kerberos errors, UNIX errors, ZERR_HEADERLEN, ZERR_ILLVAL,
ZERR_HMDEAD, ZERR_BADPKT, ZERR_VERS, ZERR_QLEN}

The ZSendList function performs identically to the ZSendNotice
function above, except that the message body is taken from the {\bf
list} argument instead of the {\bf z_message} field of {\bf notice}.  {\bf
nitems} is the number of items ({\bf nitems} $\geq$ 0) in the {\bf list}
array.  The items in {\bf list} are concatenated in order with a NULL
placed after each one.

\subsubsection{Useful Information to Include in a Notice}

The following functions provide information that it may be useful to
include in the header of a notice.

\paragraph{ZGetSender}
\label{ZGetSender}

\template{char *}{ZGetSender}{}
\etemplate
\prereq{ZInitialize}
\errors{None}

The ZGetSender function returns the name of the current user.  If
Kerberos is enabled, ZGetSender will return the Kerberos principal (in
the form {\em username\/}@{\em realm\/}).  If Kerberos is not enabled,
or the user is not authenticated, the user's username from
\filename{/etc/passwd} concatenated with `@' and the official name of
the current host (in the form {\em username\/}@{\em hostname\/}) will be
returned.

\paragraph{ZGetRealm}
\label{ZGetRealm}

\template{char *}{ZGetRealm}{}
\etemplate
\prereq{ZInitialize}
\errors{None}

The ZGetRealm function returns the realm of the current host.  If
Kerberos is enabled, ZGetRealm will return the current Kerberos realm.
Otherwise, the official name of the current host is returned.

\subsubsection{Sending Binary Data}

Occasionally it is useful to send binary data in a notice, but to
represent the binary data in printable form (for debugging or auditing
purposes).  The ZMakeAscii function makes this task easy.

\paragraph{ZMakeAscii}
\label{ZMakeAscii}

\template{Code_t}{ZMakeAscii}{buffer, buffer_len, field, field_len}
\tline{char}{*buffer}
\tline{int}{buffer_len}
\tline{unsigned char}{*field}
\tline{int}{field_len}
\etemplate
\prereq{None}
\errors{ZERR_FIELDLEN}

The ZMakeAscii function converts {\bf field_len} bytes of data
stored in {\bf field} into printable ASCII and stores them in the {\bf
buffer} of length {\bf buffer_len}.  If the converted data is longer
than {\bf buffer_len} bytes, ZERR_FIELDLEN is returned.  The
data generated by ZMakeAscii can be converted back to binary with
ZReadAscii (\myref{ZReadAscii}).

The data generated by ZMakeAscii is in the format ``{\tt 0xaabbccdd
0xeeffgghh}\ldots'', where {\tt aa} is the hexadecimal representation
of the first byte in the buffer, {\tt bb} is the second byte, etc.

\subsubsection{Sending Authenticated Notices}
\label{authenticated-notices}

A notice may be {\em authenticated\/} by the Kerberos authentication
system \cite{kerberos-paper}.
When this is done, the recipient of the notice is guaranteed that the
Kerberos principal in the {\bf z_sender} field is the actual sender
of the notice.  Note that true authentication only works when
Kerberos is enabled.  If Kerberos is not in use, any notice claiming to
be authentic will appear authentic, and there is no guarantee of actual
authenticity.

ZMakeAuthentication is the function in the \Zephyr\ library which
authenticates a packet under Kerberos.  Since this is an internal
function and should never be called directly, its template is not given
here.  The symbols ZAUTH and ZNOAUTH are defined to be
ZMakeAuthentication and NULL, respectively.  In this way one of these
abbreviations may be placed in the {\bf cert_routine} parameter for
ZFormatNotice, ZSendNotice, ZSendList or ZSrvSendList to either enable
or disable authentication.  As new forms of authentication are
developed, new abbreviations will be provided.

Authenticated notices should be used only when necessary because of
their greater overhead.  The time required to compute the DES
encryptions is usually around $\frac{1}{4}$ second at each step in the
transmission: sender, server, and recipient.

A notice can be checked for authentication with the
ZCheckAuthentication function (\myref{ZCheckAuthentication}).

\subsubsection{Sample Application}

This application demonstrates use of ZSendNotice.
It initializes the \Zephyr\ library, fills in a
notice structure (with the recipient set to the user executing the
application), and calls ZSendNotice twice to send the notice, first
unauthenticated and then authenticated.

\ss
\begin{code}
#include <zephyr/zephyr.h>

main()
{
    int status;
    ZNotice_t notice;

    /* Initialize the library */
    if ((status = ZInitialize()) != ZERR_NONE) {
        com_err("sample", status, "while initializing");
        exit(1);
    }

    bzero((char *)&notice, sizeof(notice));
    /* initialize the notice after zeroing the entire structure */
    notice.z_kind = UNACKED;    /* We don't care about acknowledgments */
    notice.z_port = 0;  /* Will be filled in by HostManager */
    notice.z_class = "MESSAGE";
    notice.z_class_inst = "PERSONAL";
    notice.z_opcode = "";
    /* The sender will get filled in by the library if 0 */
    notice.z_sender = (char *)NULL;
    notice.z_recipient = ZGetSender(); /* Send to myself */
    notice.z_default_format = "";

    notice.z_message = "Hello - This is an example!";
    /* Make sure we include the trailing NULL in the length */
    notice.z_message_len = strlen(notice.z_message)+1;

    /* First send the notice unauthenticated */
    /* ZSendNotice will automatically open a port for us */
    if ((status = ZSendNotice(&notice, ZNOAUTH)) != ZERR_NONE)
        com_err("sample", status, "while sending notice");

    /* Send it again, authenticated */
    if ((status = ZSendNotice(&notice, ZAUTH)) != ZERR_NONE) {
        com_err("sample", status, "while sending notice");
        exit(1);
    }
    exit(0);
}
\end{code}
\ess
