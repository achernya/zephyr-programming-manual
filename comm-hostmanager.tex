% Copyright (c) 1988 Massachusetts Institute of Technology
%	$Source: /tmp/git-rcsimport-Ge1GfW1M7y/rcs/comm-hostmanager.tex,v $
%	$Author: jtkohl $
%	$Header: /tmp/git-rcsimport-Ge1GfW1M7y/rcs/comm-hostmanager.tex,v 2.0 1989/04/05 15:50:13 jtkohl Exp $
%
\subsection{Communicating with the HostManager}
\label{comm-hostmanager}

The following sections describe control notices that can be
sent to a HostManager to ask it to perform various functions.

\subsubsection{Where to Send Notices}

Notices should be sent to the HostManager port on the desired host.
Since the default destination is the HostManager on the local host,
the destination need not be changed if the control message is
destined for the local host's HostManager.  Otherwise, the port may
be found by looking up port name ``zephyr-hm'', protocol ``udp'' using
the C library routine getservent().

\subsubsection{Available Commands}

The following functions are available.  They are listed in the form
[kind, class, instance, opcode].  The rest of the notice ({\em e.g.\/}
the message body) has no effect on the requested action.  Functions
which are only for internal use are not described here.

\begin{itemize}
\item {[HMCTL, HM_CTL_CLASS, HM_CTL_CLIENT, CLIENT_FLUSH]}:
Send a state-flush command to the current server.
\item {[HMCTL, HM_CTL_CLASS, HM_CTL_CLIENT, CLIENT_NEW_SERVER]}:
Find a new server.
\item {[STAT, HM_STAT_CLASS, HM_STAT_CLIENT, HM_GIMMESTATS]}: Reply with a
notice containing statistics information.
\end{itemize}

\subsubsection{Sample Application}

This application demonstrates sending a control notice to a HostManager, 
asking it to find a new server.  The application
initializes the library, fills in the control notice, and sends it off.
It does not wait for any acknowledgement.

\begin{code}
#include <zephyr/zephyr.h>

main()
{
    ZNotice_t notice;
    Code_t retval;

    /* Initialize library */
    if ((retval = ZInitialize()) != ZERR_NONE) {
        com_err("sample", retval, "while initializing");
        exit (1);
    }

    bzero((char *)&notice, sizeof(notice));
    /* initialize the notice after zeroing the entire structure */
    /* Fill in notice for a FIND-NEW-SERVER command */
    notice.z_kind = HMCTL;
    notice.z_port = 0;
    notice.z_class = HM_CTL_CLASS;
    notice.z_class_inst = HM_CTL_CLIENT;
    notice.z_opcode = CLIENT_NEW_SERVER;
    notice.z_sender = 0;
    notice.z_recipient = "";
    notice.z_default_format = "";
    notice.z_message = "";
    notice.z_message_len = 0;

    /* send it off to the local HostManager */
    if ((retval = ZSendNotice(&notice, ZNOAUTH)) != ZERR_NONE) {
        com_err("sample", retval, "while sending notice");
        exit (1);
    }
    exit(0);
}
\end{code}
