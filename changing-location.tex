% Copyright (c) 1988 Massachusetts Institute of Technology
%	$Source: /tmp/git-rcsimport-Ge1GfW1M7y/rcs/changing-location.tex,v $
%	$Author: jtkohl $
%	$Header: /tmp/git-rcsimport-Ge1GfW1M7y/rcs/changing-location.tex,v 2.0 1989/04/05 15:50:06 jtkohl Exp $
%
\subsection{Changing Your Location Information}
\label{changing-location}

The following functions allow an application to register a user,
change a user's exposure level, deregister a user, and flush all
locations associated with a certain user from the database.

\subsubsection{ZSetLocation}
\label{ZSetLocation}

\template{Code_t}{ZSetLocation}{exposure}
\tline{char}{*exposure}
\etemplate
\prereq{ZInitialize}
\errors{Kerberos errors, UNIX errors, ZERR_PKTLEN, ZERR_ILLVAL,
ZERR_HMDEAD, ZERR_BADPKT, ZERR_VERS, ZERR_QLEN, ZERR_SERVNAK,
ZERR_AUTHFAIL, ZERR_LOGINFAIL}

The ZSetLocation function registers the current user with the \Zephyr\
servers.  The information that is registered is as follows:

\begin{itemize}
\item {\bf user:} The user running the application.
\item {\bf host:} The host on which which the application is being
run.
\item {\bf tty:} The name of the terminal controlling the
application ({\it e.g.\/}, ttyp5); if the workstation is running
the X window system, the display (such as ``unix:0.0'') is used
instead.
\item {\bf time:} The current time.
\item {\bf exposure:} As indicated by the {\bf exposure} argument.
\end{itemize}

\noindent {\bf exposure} may be one of the following constants (defined in
zephyr.h):

\begin{list}{}{}
\item EXPOSE_NONE - No exposure
\item EXPOSE_OPSTAFF - Operational staff exposure
\item EXPOSE_REALMVIS - Realm visible
\item EXPOSE_REALMANN - Realm announced
\item EXPOSE_NETVIS - Net-wide visible
\item EXPOSE_NETANN - Net-wide announced
\end{list}
More detailed information about the meaning of each exposure level can
be found in the ``Exposure Levels'' section (\myref{user-location}).

If the user is already registered, this function may be used to change
the user's time and exposure information.
Note, however, that calling ZSetLocation to change the exposure may
generate spurious login notices.  If the new exposure is sufficiently broad
to allow login notifications (see \myref{exposure-levels}), a login
notice will be sent to any users who have subscribed to such notices.

This function uses an authenticated notice to present the information to
the server.

ZERR_NONE is returned if there were no errors.  ZERR_INTERNAL is
returned if any consistency checks on the acknowledgment failed.
ZERR_AUTHFAIL is returned if the server rejected the authentication
information in the notice.  ZERR_LOGINFAIL is returned if the location
information which would have been modified was not already present in
the server database.  ZERR_SERVNAK is returned if some other server
failure occured.

\subsubsection{ZUnsetLocation}
\label{ZUnsetLocation}

\template{Code_t}{ZUnsetLocation}{}
\etemplate
\prereq{ZInitialize}
\errors{Kerberos errors, UNIX errors, ZERR_PKTLEN, ZERR_ILLVAL,
ZERR_HMDEAD, ZERR_BADPKT, ZERR_VERS, ZERR_QLEN, ZERR_SERVNAK,
ZERR_AUTHFAIL, ZERR_LOGINFAIL}

The ZUnsetLocation function deletes the user's location information from
the location database.  Only the information associated with the current
host and terminal (or display) are deleted from the database.  If the
user's exposure was either EXPOSE_REALMANN or EXPOSE_NETANN, a
deregistration notice is sent by the server to the
recipients subscribed to such notifications.  The ``Exposure Levels'' section
(\myref{exposure-levels}) describes exposure levels in detail.

The errors returned and their causes are the same as described above in
ZSetLocation.

\subsubsection{ZFlushMyLocations}
\label{ZFlushMyLocations}

\template{Code_t}{ZFlushMyLocations}{}
\etemplate
\prereq{ZInitialize}
\errors{Kerberos errors, UNIX errors, ZERR_PKTLEN, ZERR_ILLVAL,
ZERR_HMDEAD, ZERR_BADPKT, ZERR_VERS, ZERR_QLEN, ZERR_SERVNAK,
ZERR_AUTHFAIL, ZERR_LOGINFAIL}

The ZFlushMyLocations function removes all location information about
the current user from the location database.  This may be used to
remove stale data from the database.  No deregistration messages are
sent.

The errors returned and their causes are the same as described above in
ZSetLocation.
