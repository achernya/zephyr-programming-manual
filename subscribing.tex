% Copyright (c) 1988 Massachusetts Institute of Technology
%	$Source: /tmp/git-rcsimport-Ge1GfW1M7y/rcs/subscribing.tex,v $
%	$Author: jtkohl $
%	$Header: /tmp/git-rcsimport-Ge1GfW1M7y/rcs/subscribing.tex,v 2.0 1989/04/05 15:57:57 jtkohl Exp $
%
\subsection{Subscribing to Notices}
\label{subscribing}

The following functions allow an application to subscribe to notices.
The subscription service is described in \myref{subscription}.

ZSubscribeTo and ZUnsubscribeTo use the ZSubscription_t structure, which
has the following fields:

\begin{itemize}
\item {\bf char *class}: The class of the subscription.
\item {\bf char *classinst}: The instance of the subscription.
\item {\bf char *recipient}: The recipient of the subscription.
\end{itemize}

\subsubsection{ZSubscribeTo}
\label{ZSubscribeTo}

\template{Code_t}{ZSubscribeTo}{sublist, nitems, port}
\tline{ZSubscription_t}{sublist[]}
\tline{int}{nitems}
\tline{unsigned short}{port}
\etemplate
\prereq{ZInitialize}
\errors{Kerberos errors, UNIX errors, ZERR_PKTLEN, ZERR_ILLVAL,
ZERR_HMDEAD, ZERR_BADPKT, ZERR_VERS, ZERR_QLEN}

The ZSubscribeTo function attempts to inform the \Zephyr\ servers that
subscriptions for the indicated {\bf port} on the current host should
be added.  The subscriptions are listed as class/instance/recipient
triples in the {\bf sublist} array.  {\bf nitems} is the number of
entries in the {\bf sublist} array.  {\bf port} will usually be the
port number of the current application.  If {\bf port} is 0, the
port number of the current application is substituted.

\subsubsection{ZUnsubscribeTo}
\label{ZUnsubscribeTo}

\template{Code_t}{ZUnsubscribeTo}{sublist, nitems, port}
\tline{ZSubscription_t}{sublist[]}
\tline{int}{nitems}
\tline{unsigned short}{port}
\etemplate
\prereq{ZInitialize}
\errors{Kerberos errors, UNIX errors, ZERR_PKTLEN, ZERR_ILLVAL,
ZERR_HMDEAD, ZERR_BADPKT, ZERR_VERS, ZERR_QLEN}

The ZUnsubscribeTo function attempts to inform the \Zephyr\ servers
that the specified subscriptions for the indicated {\bf port} on the
current host should be deleted.  The subscriptions are listed as
class/instance/recipient triples in the {\bf sublist} array.
{\bf nitems} is the number of entries in the {\bf sublist} array.
{\bf port} will usually be the port number of the current application.
If {\bf port} is 0, the port number of the current application is substituted.

ZUnsubscribeTo may be useful to remove the server default subscriptions
(\myref{default-subscriptions}),
which are automatically recorded for every port which has been passed to
the ZSubscribeTo function.  See \myref{ZRetrieveDefaultSubscriptions} to
see how to examine the default subscriptions.

\subsubsection{ZCancelSubscriptions}
\label{ZCancelSubscriptions}

\template{Code_t}{ZCancelSubscriptions}{port}
\tline{unsigned short}{port}
\etemplate
\prereq{ZInitialize}
\errors{Kerberos errors, UNIX errors, ZERR_PKTLEN, ZERR_ILLVAL,
ZERR_HMDEAD, ZERR_BADPKT, ZERR_VERS, ZERR_QLEN}

The ZCancelSubscriptions function removes {\em all\/} of the
subscriptions for the indicated {\bf port}.  If {\bf port} is 0, the
port number of the current application is substituted.

\subsubsection{Subscribing for the WindowGram Client}
\label{subscribing-zwgc}

The WindowGram client is the standard way for users to receive incoming
notices.  To accomodate this, applications will occasionally want to
subscribe to notices on behalf of the WindowGram client; in this way an
application can easily add to the types of notices that the user will
receive.  The {\bf zctl} command is an example of a program that will
subscribe on behalf of the WindowGram client.

\paragraph{ZGetWGPort}
\label{ZGetWGPort}

\template{int}{ZGetWGPort}{}
\etemplate
\prereq{None}
\errors{-1 = No port number available}

The ZGetWGPort function returns the port number associated with the
user's WindowGram client.  It does this by examining the WGFILE
environment variable, and reading the file \filename{/tmp/\$WGFILE}.
If WGFILE is not set, \filename{/tmp/wg.{\em uid\/}}, where uid is the
UNIX user ID of the user, is examined instead.  If neither file could be
found, -1 is returned.

The port number returned by ZGetWGPort can be cast to an unsigned short
value and used as the {\bf port} argument to ZSubscribeTo,
ZUnsubscribeTo, or ZCancelSubscriptions.
