% Copyright (c) 1988 Massachusetts Institute of Technology
%	$Source: /tmp/git-rcsimport-Ge1GfW1M7y/rcs/acknowledgment-format.tex,v $
%	$Author: jtkohl $
%	$Header: /tmp/git-rcsimport-Ge1GfW1M7y/rcs/acknowledgment-format.tex,v 2.0 1989/04/05 15:49:18 jtkohl Exp $
%
\subsection{Acknowledgment Structure}
\label{acknowledgment-structure}

An acknowledgment is a notice sent by either the HostManager, a
server, or a receiving client to indicate that a notice has been
received at some step in the notice relay process.  The acknowledgment
is identical to the original notice, except for the following
differences:

\begin{itemize}
\item The {\bf z_kind} field of the notice will be changed as
follows:
\begin{itemize}
\item If the notice is a HostManager-to-sending-client acknowledgment,
{\bf z_kind} will be set to HMACK.
\item If the notice is a server-to-HostManager acknowledgment,
{\bf z_kind} will be set to either SERVACK or SERVNAK depending on
whether the notice was handled successfully.
\item If the notice is a receiving-client-to-server acknowledgment,
{\bf z_kind} will be set to CLIENTACK.
\end{itemize}
\item Since the only purpose of an acknowledgment is to indicate that a
notice has been received successfully, it does not need to contain all
of the information included in the original notice.  Specifically, the
message field and the authenticator fields are left empty in most cases.
\end{itemize}

The preferred method for a client to determine which original notice
the acknowledgment is referencing is to use the {\bf z_uid} field.
The {\bf z_uid} of the acknowledgment will be the same as the {\bf
z_uid} of the original notice.  Thus this field is not a per-packet
unique ID, but a per-transaction unique ID.

When a server acknowledges a notice to the HostManager, it includes
some additional information in the message body in place of any
message that was there before.  This information takes the form of a
single, NULL-terminated string, as follows:

\begin{itemize}
\item {\bf ZSRVACK_SENT}: The notice was sent to at least one
recipient.  This does not necessarily indicate that the recipient
received the notice, but is a good indication that {\em someone\/} is
subscribing to the class and instance of the notice.
\item {\bf ZSRVACK_NOTSENT}: The notice was not sent to any
recipients.  This means that no one is subscribing to the class and
instance of the notice.
\item {\bf ZSRVACK_FAIL}: The processing of the notice failed for
some reason, most likely because of a lack of proper authentication.
This may also be caused by an attempt to send a notice to a restricted
class which the user is not authorized to send to.
\end{itemize}

More information about processing acknowledgments can be found in
\myref{receiving-acknowledgments}.
